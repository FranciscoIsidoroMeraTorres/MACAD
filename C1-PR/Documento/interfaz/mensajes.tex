\chapter{Mensajes del sistema}
\label{ch:mensajes}
\cdtLabel{apendice:Mensajes}{}

%En esta sección se describen los mensajes utilizados en el prototipo actual del sistema.
Los mensajes se refieren a todos aquellos avisos que el sistema muestra al actor para comunicar la ocurrencia de algún evento tal como un error o una operación exitosa. Estos mensajes se pueden mostrar a través de diversos canales, por ejemplo, pantalla o correo electrónico.


Cuando un mensaje es recurrente se parametrizan sus elementos, por ejemplo los mensajes: ``Aún no existen registros de \textit{Unidades Académicas} en el sistema.'', ``Aún no existen registros de \textit{Programas Académicos} en el sistema.'', 
``Aún no existen registros de \textit{alumnos} en el sistema.'', tienen una estructura similar 
por lo que, con el objetivo de que el mensaje sea genérico y pueda utilizarse en todos los casos de uso que se considere necesario, se utilizan parámetros para definir el mensaje.\\

Los parámetros también se utilizan cuando la redacción del mensaje tiene datos que son ingresados por el actor o que dependen del resultado de la operación, por ejemplo: 
``La \textit{Unidad Académica ESCOM} ha sido \textit{modificada} exitosamente.''. En este caso la redacción se presenta parametrizada de la forma: ``$<DETERMINADO> <ENTIDAD> <VALOR>$ ha sido $<OPERACION>$ exitosamente.'' y los parámetros se describen de la siguiente forma:

\begin{itemize}
	\item DETERMINADO ENTIDAD: Es un artículo determinado más el nombre de la entidad sobre la cual se realizó la acción.
	\item VALOR: Es el valor asignado al atributo de la entidad, generalmente es el nombre o la clave.
	\item OPERACIÓN: Es la acción que el actor solicitó realizar.
\end{itemize}

En el ejemplo anterior se hace referencia a $<VALOR>$, es decir: \textit{ESCOM} es el \textbf{valor} de la \textbf{entidad} \textit{escuela}. Cada mensaje enlista los parámetros  que utiliza, sin embargo aquí se definen los más comunes a fin de simplificar la descripción de los mensajes:

\begin{description}
	\item [$<ARTICULO>$:] Se refiere a un {\em artículo} el cual puede ser DETERMINADO (El $\mid$ La $\mid$ Lo $\mid$ Los $\mid$ Las) o INDETERMINADO (Un $\mid$ Una $\mid$ 
	Uno $\mid$ Unos $\mid$Unas) se aplica generalmente sobre una ENTIDAD, ATRIBUTO o VALOR.
	
	\item [$<CAMPO>$:] Se refiere a un campo del formulario. Por lo regular es el nombre de un atributo en una entidad.
	
	\item [$<CONDICION>$:] Define una expresión booleana cuyo resultado deriva en {\em falso} o {\em verdadero} y suele ser la causa del mensaje.
	
	\item [$<DATO>$:] Es un sustantivo y generalmente se refiere a un atributo de una entidad descrito en el modelo estructural del negocio, por ejemplo: nombre de la unidad de aprendizaje, nombre del alumno, RFC del profesor, etc. %ATRIBUTO
	
	\item [$<ENTIDAD>$:] Es un sustantivo y generalmente se refiere a una entidad del modelo estructural del negocio, por ejemplo: programa académico, alumno, profesor, etc.
	\item [$<OPERACION>$:] Se refiere a una acción que se debe realizar sobre los datos de una o varias entidades. Por ejemplo: registrar, eliminar, actualizar, por mencionar algunos. Comúnmente la OPERACIÓN va concatenada con el sustantivo, por ejemplo: Registro de un nuevo beneficio, registro de una actividad, eliminar una tarea y demás.
	
	\item [$<VALOR>$:] Es un sustantivo concreto y generalmente se refiere a un valor en específico. Por ejemplo: \textit{Histología I} es un \textbf{valor} concreto de la \textbf{entidad} \textit{Unidad de Aprendizaje}.
	
\end{description}
%___________________________Plantilla_______________________________
%===========================  MSGX ==================================
%\begin{mensaje}{MSGX}{}{}
%	\item[Canal:] 
%	\item[Propósito:] 
%	\item[Redacción:]
%	\item[Parámetros:] 
%	\begin{itemize}
%		\item 
%	\end{itemize}
%	\item[Ejemplo:]  
%	\item[Referenciado por: ]
%\end{mensaje}

%===========================  MSG1 ==================================
\begin{mensaje}{MSG1}{Operación Exitosa}{Informativo }
	\item[Canal:] Sistema
	\item[Propósito:] Notificar al actor que la operación solicitada al sistema se llevó a cabo exitosamente.
	\item[Redacción:] La $<OPERACION>$ se llevo a cabo correctamente.
	\item[Parámetros:] OPERACION: Actividad que el actor debe realizar.
	\item[Ejemplo:] La reincripción se llevo a cabo correctamente.
	\item[Referenciado por:] \refIdElem{HR-UA-CU2}
\end{mensaje}

%============================== MSG2 =================================
\begin{mensaje}{MSG2}{Operación Fallida}{Error}
	\item[Canal:] Sistema
	\item[Propósito:] Notificar al actor que la operación solicitada al sistema no pudo llevarse a cabo.
	\item[Redacción:] Error, $<ERROR_ENCONTRADO>$.
	\item[Parámetros:] ERROR\_ENCONTRADO: Error que el sistema identificó al intentar realizar la operación.
	\item[Ejemplo:] Error, No hay conexión al sistema.(Base de Datos)
	\item[Referenciado por: ] 
\end{mensaje}


%===========================  MSG3 ==================================
\begin{mensaje}{MSG3}{Elementos No Disponibles}{Error}
	\item[Canal:] Sistema
	\item[Propósito:] Notificar al actor que no existen elementos en el sistema por mostrar.
	\item[Redacción:] ¡No hay $<ELEMENTOS>$ para mostrarte!
	\item[Parámetros:] ELEMENTOS: Información de la cual se requiere para concluir un proceso
	\item[Ejemplo:] ¡No hay registros de aspirantes para mostrarte!
	\item[Referenciado por: ] 
	
\end{mensaje}

%============================== MSG4 =================================
\begin{mensaje}{MSG4}{Elemento No Disponible}{Error}
	\item[Canal:] Sistema
	\item[Propósito:] Notificar al actor que el elemento que se desea ver no existe o no está disponible en el sistema.
	\item[Redacción:] ¡Este/Esta $<ELEMENTO>$ no esta disponible aún!
	\item[Parámetros:] ELEMENTO:Información de un registro en específico de la cual se requiere para concluir un proceso.
	\item[Ejemplo:] ¡Esta boleta no esta disponible aún!
	\item[Referenciado por: ] 
\end{mensaje}

%============================== MSG5 =================================
\begin{mensaje}{MSG5}{Duplicado de Información}{Error}
	\item[Canal:] Sistema
	\item[Propósito:] Notificar al actor que ya existe un elemento con la misma información que desea ingresar.
	\item[Redacción:] ¡Ya existe un/a $<ELEMENTO>$ con la misma información!
	\item[Parámetros:] ELEMENTO: Parte de la información que esta duplicada
	\item[Ejemplo:]¡Ya existe una boleta con la misma información!
	\item[Referenciado por: ] 
\end{mensaje}

%============================== MSG6 =================================
\begin{mensaje}{MSG6}{Falta dato obligatorio}{Error}
	\item[Canal:] Sistema
	\item[Propósito:] Notificar al actor la omisión de algún dato obligatorio por ingresar.
	\item[Redacción ] !Error¡ El campo $<DATO>$ es un campo obligatorio, favor de ingresar el campo faltante
	\item[Parámetros:] DATO: Información que se requiere para completar la operación.
	\item[Ejemplo:] !ERROR¡ El campo Referencia Bibliografíca es un campo obligatorio, favor de ingresar el campo faltante.
	\item[Referenciado por: ] 
\end{mensaje}

%============================== MSG7 =================================
\begin{mensaje}{MSG7}{Formato de campo Incorrecto}{Error }
	\item[Canal:] Sistema
	\item[Propósito:] Notificar al actor que el dato ingresado en alguno de los campos del formulario no cumple con el tipo de dato definido en el diccionario de datos.
	\item[Redacción:] !Error¡ El campo $<DATO>$ ingresado es incorrecto, favor de ingresar un dato válido.
	\item[Parámetros:] DATO: Información que se requiere para completar la operación.
	\item[Ejemplo:] !ERROR¡ El campo Unidad de Aprendizaje ingresado es incorrecto, favor de ingresar un dato válido
	\item[Referenciado por: ] 
\end{mensaje}


%===============================MSG8=============================

\begin{mensaje}{MSG8}{Operación por Atender}{Informativo }
	\item[Canal:] Correo / SMS / WhatsApp
	\item[Propósito:] Notificar al actor que existen operaciones que requieren de su atención o supervisión para llevarse a cabo.
	\item[Redacción:]Calmécac te notifica que hay $<ELEMENTOS>$ que requieren de tu atención.
	\item[Parámetros:]ELEMENTOS: Aquella información que el actor debe supervisar o atender.
	\item[Ejemplo:] Calmécac te notifica que hay registros de aspirantes que requieren de tu atención.
	\item[Referenciado por: ] 	
\end{mensaje}

%===============================MSG9=============================

\begin{mensaje}{MSG9}{Límite de Tiempo}{Informativo }
	\item[Canal:] Correo / SMS / WhatsApp
	\item[Propósito:] Notificar al actor que el límite de tiempo para que realice una operación está por agotarse.
	\item[Redacción:]¡ALERTA! Recuerda que debes realizar la $<OPERACION>$ antes del $<TIEMPO>$
	\item[Parámetros:]OPERACION: Tarea a realizar por el actor
	TIEMPO: Fecha para la cual el límite 
	\item[Ejemplo:] ¡ALERTA! Recuerda que debes realizar la reinscripción antes del 10 de Mayo del 2017.
	\item[Referenciado por: ] 	
\end{mensaje}


%===============================MSG10=============================
\begin{mensaje}{MSG10}{Cierre de Operación}{Informativo}
	\item[Canal:] Correo / SMS / WhatsApp
	\item[Propósito:] Notificar al actor que el límite de tiempo para que realice la operación se ha agotado y sugiere realizar algo para poder llevarla a cabo.
	\item[Redacción:] ¡ALERTA! El tiempo para realizar la $<OPERACION>$ se ha acabado. Calmécac te sugiere $<SUGERENCIA>$.
	\item[Parámetros:] 
	\begin{itemize}
		\item OPERACION: Tarea a realizar por el actor
		\item SUGERENCIA: Proceso u operación que el actor debe realizar para poder concluir una actividad.
	\end{itemize}
	\item[Ejemplo:] ¡ALERTA! El tiempo para realizar la reinscripción se ha acabado. Calmécac te sugiere ir al Departamento de Gestión Escolar para poder reinscribirte.
	\item[Referenciado por: ] 
\end{mensaje}

%===========================  MSG11 ==================================
\begin{mensaje}{MSG11}{Recomendación para revisar configuración de servicios web}{Informativo}
	\item[Canal:] Correo / SMS / WhatsApp
	\item[Propósito:] Notificar al actor que la es necesario revisar la configuración actual de los servicios web, ya que se acerca un proceso de carga de datos al sistema.
	\item[Redacción:] Recomendación del Calmécac. La $<OPERACION>$ ha concluido, y se sugiere que se revise la configuración de los servicios web del Calmécac.
	\item[Parámetros:] OPERACION: Tarea realizada por el Politécnico.
	\item[Ejemplo:] Recomendación del Calmecac. La generación de citas ha concluido y se sugiere que se revise la configuración de los servicios web del Calmécac.
	\item[Referenciado por: ] 	
\end{mensaje}


%===========================  MSG12 ==================================
\begin{mensaje}{MSG12}{Confirmar operación sin cambios posteriores}{Confirmación}
	\item[Canal:] Sistema
	\item[Propósito:] Solicitar al actor la confirmación de una operación de impacto que no podrá ser revertida.
	\item[Redacción:] ¿Desea confirmar la finalización $<OPERACION>$? La operación no podrá ser revertida una vez confirmada.
	Calmécac te notifica que es necesaria tú aprobación para concluir la $<OPERACION>$
	\item[Parámetros:] OPERACION: Aquella operación que el actor debe confirmar.
	\item[Ejemplo:] ¿Desea confirmar la finalización del registro de aspirantes? La operación no podrá ser revertida una vez confirmada.
	Calmécac te notifica que es necesario tú aprobación para concluir el registro de aspirantes
	\item[Referenciado por: ] 
\end{mensaje}

%===========================  MSG13 ==================================
\begin{mensaje}{MSG13}{Operación en Curso}{Informativo}
	\item[Canal:] Sistema
	\item[Propósito:] Notificar al actor que el sistema esta actualmente ejecutando una transacción.
	\item[Redacción:] Calmécac te notifica que se esta ejecutando la/el $<TRANSACCION>$
	\item[Parámetros:] TRASACCION: Aquella transaccion que se esta ejecutando actualmente
	\item[Ejemplo:] Calmécac te notifica que se esta ejecutando la generación de citas de reinscripción.
	\item[Referenciado por: ]     
\end{mensaje}


%===========================  MSG14 ==================================
\begin{mensaje}{MSG14}{No tienes notificaciones por atender}{Informativo}
	\item[Canal:] Correo / SMS / WhatsApp
	\item[Propósito:] Notificar al actor que no tiene mensajes o notificaciones que requieran de su atención.
	\item[Redacción:] Calmécac te notifica que no tienes mensajes o notificaciones pendientes que requieran de tu atención
\end{mensaje}

%===========================  MSG15 ==================================
\begin{mensaje}{MSG15}{No existe información en el sistema}{Error}
	\item[Canal:] Sistema
	\item[Propósito:] Notificar al actor que no se puede llevar a cabo la operación solicitada pues no existe la información suficiente en el sistema.
	\item[Redacción:] !Error! Falta información necesaria sobre $<INFORMACION>$ en el sistema para poder realizar la/el $<OPERACION>$ \\
	\item[Parámetros:] 
	\begin{Titemize}
		\Titem INFORMACIÓN: El tipo de información que hace falta en el sistema.
		\Titem OPERACION: Tarea a realizar por el actor
	\end{Titemize}
	\item[Ejemplo:] ¡Error! Falta información necesaria sobre Modalidad Educativa en el sistema para poder realizar el registro de Unidades de Aprendizaje
	\item[Referenciado por: ] \refIdElem{HR-UA-CU4}
\end{mensaje}

%===========================  MSG16 ==================================
\begin{mensaje}{MSG16}{Condiciones necesarias para realizar operación}{Error}
	\item[Canal:] Sistema
	\item[Propósito:] Notificar al actor que no se puede realizar la operación solicitada debido a que no se cumplen las condiciones necesarias para hacerlo
	\item[Redacción:] No es posible $<OPERACION> <ARTICULO\_ENTIDAD>$ pues es necesario que $<CONDICIONES>$
	\item[Parámetros:] 
	\begin{itemize}
		\item OPERACION: Es la operación que el actor intentó llevar a cabo.
		\item ARTICULO\_ENTIDAD: Es la entidad sobre la cuál se intentó realizar la operación acompañado del artículo de género correspondiente.
		\item CONDICIONES: Son las condiciones necesarias que no se satisfacieron.
	\end{itemize}
	\item[Ejemplo:] No es posible activar el plan de estudio pues es necesario que se especifiquen sus unidades de aprendizaje.
	\item[Referenciado por: ] 
	
\end{mensaje}

%===========================  MSG17 ==================================
\begin{mensaje}{MSG17}{Eliminar elemento}{Confirmación}
	\item[Canal:] Sistema
	\item[Propósito:] Solicitar la confirmación del actor para la eliminación de un elemento.
	\item[Redacción:] ¿Desea eliminar $<ARTICULO>$ + $<ELEMENTO>$ + $<SELECCION>$?
	\item[Parámetros:] 
	\begin{itemize}
		\item ARTICULO: Es la parte de la oración que se ocupa de expresar el género (masculino/femenino).
		\item ELEMENTO: Es el elemento que se requiere eliminar.
		\item SELECCION:Dependiendo del género en la oración se ocupa, seleccionado(masculino)  ó seleccionada(femenino).
	\end{itemize}
	\item[Ejemplo:] ¿Desea eliminar el ciclo escolar seleccionado?
	\item[Referenciado por: ] 
\end{mensaje}

%===========================  MSG18 ==================================
\begin{mensaje}{MSG18}{Relación de fechas incorrecta}{Error}
	\item[Canal:] Sistema
	\item[Propósito:] Notificar al actor que la relación entre dos fechas ingresadas no es la esperada.
	\item[Redacción:] La $<FECHA1>$ de $<ENTIDAD1>$ debe ser $<CONDICION>$ $<FECHA2>$ de $<ENTIDAD2>$
	\item[Parámetros:] 
	\begin{itemize}
		\item FECHA1: Fecha de una entidad.
		\item ENTIDAD1: Entidad que requiere de una fecha.
		\item CONDICION: La condición de relación que deben satisfacer FECHA1 Y FECHA2.
		\item FECHA2: Fecha de una entidad.
		\item ENTIDAD2: Entidad que requiere de una fecha.
	\end{itemize}
	\item[Ejemplo:] La fecha de fin de actividad debe ser menor que la fecha de fin de período escolar.
	\item[Referenciado por: ] 
\end{mensaje}

%===========================  MSG19 ==================================
\begin{mensaje}{MSG19}{Relación de cantidades incorrecta}{Error}
	\item[Canal:] Sistema
	\item[Propósito:] Notificar al actor que la relación entre dos cantidades ingresadas no es la esperada.
	\item[Redacción:] La $<CANTIDAD1>$ debe ser $<CONDICION>$ $<CANTIDAD2>$.
	\item[Parámetros:] 
	\begin{itemize}
		\item CANTIDAD1: Cantidad de un elemento.
		\item CONDICION: La condición de relación que deben satisfacer CANTIDAD1 Y CANTIDAD2.
		\item CANTIDAD2: Cantidad de un elemento.
	\end{itemize}
	\item[Ejemplo:] La carga mínima debe ser menor que la carga media.
	\item[Referenciado por: ] 
\end{mensaje}

%===========================  MSG20 ==================================
\begin{mensaje}{MSG20}{Relación de horas incorrecta}{Error}
	\item[Canal:] Sistema
	\item[Propósito:] Notificar al actor que la relación entre dos horas ingresadas no es la esperada.
	\item[Redacción:] Las(s) $<HORAS1>$ de $<ENTIDAD1>$ debe(n) ser $<CONDICION>$ $<HORAS2>$ de $<ENTIDAD2>$
	\item[Parámetros:] 
	\begin{itemize}
		\item HORAS1: Horas asociadas a una entidad.
		\item ENTIDAD1: Entidad que requiere de una asginación de horas.
		\item CONDICION: La condición de relación que deben satisfacer HORAS1 Y HORAS2.
		\item HORAS2: Horas asociadas a una entidad.
		\item ENTIDAD2: Entidad que requiere de una asginación de horas.
	\end{itemize}
	\item[Ejemplo:] Las horas de un contenido deben ser menores que las horas de la unidad de aprendizaje.
	\item[Referenciado por: ] 
\end{mensaje}


%===========================  MSG21 ==================================
\begin{mensaje}{MSG21}{Elementos mínimos necesarios}{Error}
	\item[Canal:] Sistema
	\item[Propósito:] Notificar al actor que debe registrar al menos cierta cantidad de elementos para realizar la operación.
	\item[Redacción:] Es necesario $<OPERACION>$ al menos $<CANTIDAD>$ $<ENTIDAD>$.
	\item[Parámetros:] 
	\begin{itemize}
		\item OPERACIÓN: Operación que debe realizar el usuario. Puede ser registrar o seleccionar.
		\item CANTIDAD: La cantidad necesaria de elementos que el actor debe registrar.
		\item ENTIDAD: Es la entidad de la cuál se debe realizar la acción.
	\end{itemize}
	\item[Ejemplo:]  Es necesario registrar al menos un plan de estudio.
	\item[Referenciado por: ] 
\end{mensaje}


%===========================  MSG22 ==================================
\begin{mensaje}{MSG22}{Notificación de operación.}{Informativo}
	\item[Canal:] Sistema
	\item[Propósito:] Notificar al actor la realización de una operación.
	\item[Redacción:] $<ARTICULO\_ENTIDAD>$ $<NOMBRE>$ ha sido $<OPERACION>$
	\item[Parámetros:] 
	\begin{itemize}
		\item ARTICULO\_ENTIDAD: Es la entidad a la que pertenece el elemento sobre el que se realizó la operación..
		\item NOMBRE: Es el nombre del elemento.
		\item OPERACIÓN: Es la operación que ha sido realizada.
	\end{itemize}
	\item[Ejemplo:] El período escolar 2017-2018/3-A1 ha sido eliminado.
	\item[Referenciado por: ] 
\end{mensaje}

%===========================  MSG23 ==================================
\begin{mensaje}{MSG23}{No es posible realizar operación sobre elemento}{Informativo}
	\item[Canal:] Sistema
	\item[Propósito:] Notificar al actor que no es posible realizar una operación sobre un elemento debido a la existencia de otros elementos asociados a él.
	\item[Redacción:] No es posible $<OPERACION>$ $<ARTICULO\_ENTIDAD>$ $<NOMBRE>$ debido a que tiene $<ELEMENTOS\_ASOCIADOS>$ asociados.
	\item[Parámetros:] 
	\begin{itemize}
		\item OPERACIÓN: La operación que no se puede llevar a cabo.
		\item ARTICULO\_ENTIDAD: Es la entidad de la cuál se quiere eliminar un elemento.
		\item NOMBRE: Es el nombre del elemento.
		\item ELEMENTOS\_ASOCIADOS: Son los elementos que se encuentran asociados al elemento que se quiere eliminar.
	\end{itemize}
	\item[Ejemplo:] No es posible eliminar el edificio Edificio Inteligente debido a que tiene espacios asociados. 
	\item[Referenciado por: ] 
\end{mensaje}

%===========================  MSG24 ==================================
\begin{mensaje}{MSG24}{Confirmación de eliminación en cadena}{Confirmación}
	\item[Canal:] Sistema
	\item[Propósito:] Solicitar al actor la confirmación para eliminar un elemento que tiene como consecuencia la elimación de otros elementos.
	\item[Redacción:] Al eliminar $<ARTICULO\_ENTIDAD>$ $<NOMBRE>$ se eliminarán $<ARTICULO\_ENTIDAD\_ASOCIADA>$ que se listan a continuación, ¿desea continuar?
	\item[Parámetros:] 
	\begin{itemize}
		\item ARTICULO\_ENTIDAD: Es la entidad de la cuál se eliminará el elemento.
		\item NOMBRE: Es el identificador del elemento.
		\item ARTICULO\_ENTIDAD\_ASOCIADA: Es la entidad de las cuál se eliminarán elementos debido a la eliminación de NOMBRE.
	\end{itemize}
	\item[Ejemplo:]  Al eliminar el periodo escolar se eliminarán las estructuras académicas que se listan a continuación, ¿desea continuar?
	\item[Referenciado por: ] 
\end{mensaje}

%===========================  MSG25 ==================================
\begin{mensaje}{MSG25}{Confirmación de operación}{Confirmación}
	\item[Canal:] Sistema
	\item[Propósito:] Solicitar al actor la confirmación para llevar a cabo una operación.
	\item[Redacción:] ¿Desea $<OPERACION>$ $<ARTICULO\_ENTIDAD>$ $<NOMBRE>$?
	\item[Parámetros:] 
	\begin{itemize}
		\item OPERACIÓN: Es la operación que el actor desea realizar.
		\item ARTICULO\_ENTIDAD: Es la entidad de la cuál se eliminará el elemento.
		\item NOMBRE: Es el identificador del elemento.
	\end{itemize}
	\item[Ejemplo:] ¿Desea editar el plan de estudio Plan 2009?
	\item[Referenciado por: ] 
\end{mensaje}

%===========================  MSG26 ==================================
\begin{mensaje}{MSG26}{El período escolar ya inició}{Informativo}
	\item[Canal:] Sistema
	\item[Propósito:] Notificar al actor la inhabilitación para edición de ciertas características del período escolar debido al comienzo del mismo.
	\item[Redacción:] El período escolar ha comenzado por lo tanto existen campos que no es posible editar.
	\item[Referenciado por: ]
\end{mensaje}

%===========================  MSG27 ==================================
\begin{mensaje}{MSG27}{Usuario y/o contraseña incorrectos}{Error}
	\item[Canal:] Sistema
	\item[Propósito:] Notificar al actor que el usuario o contraseña que ingresó son incorrectos.
	\item[Redacción:] ¡Error! el usuario o contraseña ingresada son incorrectos, favor de ingresarlos nuevamente.
\end{mensaje}

%===========================  MSG28 =================================
\begin{mensaje}{MSG28}{Usuario Inhabilitado}{Informativo}
	\item[Canal:] Sistema
	\item[Propósito:] Notificar al actor que su usuario ha quedado inhabilitado y que requiere contactar al administrador.
	\item[Redacción:] Calmécac te notifica que tu usuario ha quedado inhabilitado y te recomienda contactar al administrador.
\end{mensaje}

%===========================  MSG29 ==================================
\begin{mensaje}{MSG29}{Usuario bloqueado}{Informativo}
	\item[Canal:] Sistema
	\item[Propósito:] Notificar al actor que su usuario ha quedado bloqueado por rebasar el límite de intentos permitidos por haber ingresado su usuario o contraseña incorrectamente.
	\item[Redacción:] Calmécac te notifica que tu usuario ha quedado bloqueado por rebasar el límite de intentos permitidos.
\end{mensaje}

%===========================  MSG30 ==================================
\begin{mensaje}{MSG30}{Bienvenida}{Informativo}
	\item[Canal:] Sistema
	\item[Propósito:] Mostrar un mensaje de bienvienida al actor
	\item[Redacción:] Calmécac te da la bienvenida $<USUARIO>$
	\item[Parámetros:] USUARIO: Nombre del usuario que inicia sesión
	\item[Ejemplo:] Calmécac te da la bienvenida Itzel Torres
\end{mensaje}

%===========================  MSG31 ==================================
\begin{mensaje}{MSG31}{Vuelva Pronto}{Informativo}
	\item[Canal:] Sistema
	\item[Propósito:] Mostrar un mensaje de despedida al actor
	\item[Redacción:] Vuelva pronto $<USUARIO>$
	\item[Parámetros:] USUARIO: Nombre del usuario que inicia sesión
	\item[Ejemplo:] Vuelve pronto Elizabeth Mendieta
\end{mensaje}


%===========================  MSG32 ==================================
\begin{mensaje}{MSG32}{Cancelación de préstamo}{Confirmación}
	\item[Canal:] Sistema
	\item[Propósito:] Solicitar la confirmación del actor para la cancelación de un préstamo de espacios.
	\item[Redacción:] ¿Está seguro que desea cancelar este préstamo?
	\item[Referenciado por: ] 
\end{mensaje}

%=========================== MSG33 ==================================
\begin{mensaje}{MSG33}{No es posible cancelar el préstamo}{Error}
	\item[Canal:] Sistema
	\item[Propósito:] Notificar al actor que no es posible cancelar el préstamo debido a que el espacio es usado en la estructura académica del la Unidad de Aprendizaje.
	\item[Redacción:] No es posible cancelar el préstamo debido a que la Unidad Académica prestataria hace uso del mismo dentro de su estructura académica.
	\item[Referenciado por: ] 
\end{mensaje}

%===========================  MSG34 ==================================
\begin{mensaje}{MSG34}{Alerta de eliminación de bibliografía}{Alerta}
	\item[Canal:] Sistema
	\item[Propósito:] Notificar al actor que la bibliografía se eliminará únicamente de su Unidad de Aprendizaje pues existen otras Unidades de Aprendizaje que la usan.
	\item[Redacción:] Esta bibliografía es usada por otras Unidades de Aprendizaje por lo tanto se eliminará solamente de esta Unidad de Aprendizaje.
	\item[Referenciado por: ] 
\end{mensaje}

%===========================  MSG35 ==================================
\begin{mensaje}{MSG35}{Advertencia sobre duración de período escolar}{Alerta}
	\item[Canal:] Sistema
	\item[Propósito:] Notificar al actor que la duración del período escolar introducida no es la recomendada.
	\item[Redacción:] La duración del período escolar  son $<DURACION>$. La duración recomendada para un período escolar modalidad $<MODALIDAD>$ es de $<DURACION_MINIMA>$ a $<DURACION_MAXIMA>$ meses.
	\item[Parámetros:] 
	\begin{itemize}
		\item DURACION: Es la duración del período escolar.
		\item MODALIDAD: Es la modalidad del período escolar.
		\item DURACION\_MINIMA: Es la duración mínima recomendada del período escolar de acuerdo con su modalidad.
		\item DURACION\_MAXIMA: Es la duración máxima recomendada del período escolar de acuerdo con su modalidad.
	\end{itemize}
	\item[Ejemplo:] La duración del período escolar son 3 meses. La duración recomendda para un período escolar modalidad escolarizada es de cinco a seis meses.
	\item[Referenciado por:] 
\end{mensaje}

%===========================  MSG36 ==================================
\begin{mensaje}{MSG36}{Traslape de períodos escolares}{Error}
	\item[Canal:] Sistema
	\item[Propósito:] Notificar al actor que el período escolar introducido se traslapa con un período escolar ya existente.
	\item[Redacción:] El período escolar se traslapa con un período escolar existe. Por favor revise las fechas de inicio y término.
	\item[Referenciado por:] 
\end{mensaje}


%===========================  MSG37 ==================================
\begin{mensaje}{MSG37}{Inconsistencia de fechas de actividad con su tipo}{Error}
	\item[Canal:] Sistema
	\item[Propósito:] Notificar al actor que las fechas de una actividad no son congruentes con su tipo.
	\item[Redacción:] $<ARTICULO\_ACTIVIDAD>$ se dijo $<RELACION>$ del período escolar pero la actividad no se encuentra $<RELACION>$ del período escolar.
	\item[Parámetros:] 
	\begin{itemize}
		\item ARTICULO\_ACTIVIDAD: La actividad cuyas fechas no cumplen con lo establecido acompañado del artículo de género correspondiente.
		\item RELACIÓN: La relación que debe existir entre la actividad y el período. Puede ser dentro o fuera.
	\end{itemize}
	\item[Ejemplo:] La celebración del día del politécnico se dijo dentro del período escolar  pero la actividad no se encuentra dentro del período escolar.
	\item[Referenciado por:] 
\end{mensaje}

%===========================  MSG38 ==================================
\begin{mensaje}{MSG38}{Duración exacta de período}{Error}
	\item[Canal:] Sistema
	\item[Propósito:] Notificar al actor que la duración del período no cumple con la duración establecida.
	\item[Redacción:] La duración de $<ARTICULO\_PERIODO>$  debe ser $<DURACION>$ $<UNIDAD\_TIEMPO>$.
	\item[Parámetros:] 
	\begin{itemize}
		\item ARTICULO\_PERIODO: Es el período a cuya duración se refiere el mensaje acompañado del artículo de género correspondiente.
		\item DURACION: Es la duración que debe tener el período.
		\item UNIDAD\_TIEMPO: Es la unidad de tiempo en la que se mide la duración del período.
	\end{itemize}
	\item[Ejemplo:] La duración de la semana de inducción debe ser 1 semana.
	\item[Referenciado por:] 
\end{mensaje}


% Mensaje disponible
%===========================  MSG39 ==================================


%===========================  MSG40 ==================================
\begin{mensaje}{MSG40}{Inhabilitar usuario}{Confirmación}
	\item[Canal:] Sistema
	\item[Propósito:] Solicitar al actor la confirmación para la inhabilitación de un usuario.
	\item[Redacción:] El usuario quedará inactivo y no podrá usar el sistema hasta que vuelva a estar activo. ¿Desea continuar? \item[Referenciado por:] 
\end{mensaje}

%===========================  MSG41 ==================================
\begin{mensaje}{MSG41}{Habilitar usuario}{Confirmación}
	\item[Canal:] Sistema
	\item[Propósito:] Solicitar la confirmación del actor para la habilitación de un usario bajo cierto perfil
	\item[Redacción:] El usuario quedará activo con el perfil $<PERFIL>$ y podrá usar el sistema. ¿Desea continuar?
	\item[Parámetros:] PERFIL: El perfil bajo el que el usuario quedará activo.
	\item[Ejemplo:] El usuario quedará activo con el perfil Jefe de Gestión Académica y podrá usar el sistema. ¿Desea continuar?
	\item[Referenciado por:] 
\end{mensaje}

%===========================  MSG42 ==================================
\begin{mensaje}{MSG42}{Confirmación de operación con errores}{Confirmación}
	\item[Canal:] Sistema
	\item[Propósito:] Solicitar la confirmación del actor para realizar una operación de la que se encontraron errores en sus datos.
	\item[Redacción:] Existen errores en los datos necesarios para realizar la operación. ¿Desea continuar?
	\item[Referenciado por:] %\refIdElem{DAECU3.1}
\end{mensaje}

%===========================  MSG43 ==================================
\begin{mensaje}{MSG43}{Elemento listo para revisión}{Notificación}
	\item[Canal:] Sistema
	\item[Propósito:] Notificar al actor que existe un elemento  que requiere su revisión.
	\item[Redacción:] $<ENTIDAD> <NOMBRE>$ listo para revisión. 
	\item[Parámetros:] 
	\begin{itemize}
		\item ENTIDAD: Es la entidad a la que pertence el elemento que necesita revisión.
		\item NOMBRE: Es el nombre de la entidad a la que pertenece el elemento que necesita revisión.
	\end{itemize}
	\item[Ejemplo:] Plan de estudio Plan 2009 listo para revisión.
	\item[Referenciado por:] 
\end{mensaje}

%===========================  MSG44 ==================================
\begin{mensaje}{MSG44}{Solicitud de cambios}{Notificación}
	\item[Canal:] Sistema
	\item[Propósito:] Notificar al actor la existencia de una solicitud de cambios a un elemento.
	\item[Redacción:] Solicitud de cambios en $<ENTIDAD> <NOMBRE>$
	\item[Parámetros:] 
	\begin{itemize}
		\item ENTIDAD: Es la entidad a la que pertence el elemento que tiene una solicitud de cambios.
		\item NOMBRE: Es el nombre de la entidad a la que pertenece el elemento que tiene una solicitud de cambios.
	\end{itemize}
	\item[Ejemplo:] Solicitud de cambios en Unidad de Aprendizaje Economía I.
	\item[Referenciado por:] 
\end{mensaje}

%===========================  MSG45 ==================================
\begin{mensaje}{MSG45}{Elemento aprobado, operación necesaria}{Notificación}
	\item[Canal:] Sistema
	\item[Propósito:] Notificar al actor que un elemento ha sido aprobado por lo que es necesario la realización de una operación.
	\item[Redacción:] $<ENTIDAD> <NOMBRE>$ ha sido aprobado, por favor $<OPERACION\_NECESARIA>$. 
	\item[Parámetros:] 
	\begin{itemize}
		\item ENTIDAD: Es la entidad a la que pertence el elemento que ha sido aprobado.
		\item NOMBRE: Es el nombre de la entidad a la que pertenece el elemento que ha sido aprobado.
		\item OPERACIÓN\_NECESARIA: Es la operación que se solicita realizar al actor.
	\end{itemize}
	\item[Ejemplo:] Programa Académico Ing. en Sistemas Automotrices ha sido aprobado, por favor registre el plan de estudio asociado.
	\item[Referenciado por:] 
\end{mensaje}

%===========================  MSG46 ==================================
\begin{mensaje}{MSG46}{Cambio de estado en elemento}{Notificación}
	\item[Canal:] Sistema
	\item[Propósito:] Notificar al actor de un cambio de estado en un elemento.
	\item[Redacción:] $<ENTIDAD> <NOMBRE>$ ha sido $<CAMBIO\_ESTADO>$
	\item[Parámetros:] 
	\begin{itemize}
		\item ENTIDAD: Es la entidad a la que pertence el elemento que ha cambiado de estado.
		\item NOMBRE: Es el nombre de la entidad a la que pertenece el elemento que ha cambiado de estado.
		\item CAMBIO\_ESTADO: Es el cambio de estado que ha sucedido con el elemento. 
	\end{itemize}
	\item[Ejemplo:] Plan de Estudios Plan 2012 ha sido puesto en vigor.
	\item[Referenciado por:] 
\end{mensaje}

%===========================  MSG47 ==================================
\begin{mensaje}{MSG47}{Longitud incorrecta}{Error}
	\item[Canal:] Sistema
	\item[Propósito:] Notificar al actor que la longitud del dato ingresado no cumple con lo definido en el modelo de información. 
	\item[Redacción:] La longitud ingresada es incorrecta.
	%\item[Parámetros:] 
	%\item[Ejemplo:] La fecha de autorización de un plan de estudios debe ser igual o anterior a la fecha actual.
	\item[Referenciado por: ] 
\end{mensaje}


%===========================  MSG48 ==================================
\begin{mensaje}{MSG48}{No existe información registrada para elemento}{Error}
	\item[Canal:] Sistema
	\item[Propósito:] Notificar al actor que no extiste información registrada en sistema relacionada a algún parámetro en específico.
	\item[Redacción:] No existe información de $Art1$ $Var1$ registrada para $Art2$ $Var2$.
	\item[Parámetros:] 
	\begin{itemize}
		\item Art1: Artículo que define el género de la entidad $Var1$
		\item Var1: Entidad de la cual no existe información registrada
		\item Art2: Artículo que define el género de la entidad $Var2$
		\item Var2: Entidad a la cual está asociada la entidad $Var2$
	\end{itemize}
	\item[Ejemplo:] No existe información de la Unidad de Aprendizaje registrada en sistema relacionada para el Plan de Estudio.
	\item[Referenciado por: ] 
\end{mensaje}

%===========================  MSG49 ==================================
\begin{mensaje}{MSG49}{Relación de fecha y referencia incorrecta}{Error}
	\item[Canal:] Sistema
	\item[Propósito:] Notificar al actor que la relación entre una fecha ingresada y una fecha de referencia no es la esperada.
	\item[Redacción:] La $<FECHA1>$ de $<ENTIDAD>$ debe ser $<CONDICION>$ $<FECHA2>$
	\item[Parámetros:] 
	\begin{itemize}
		\item FECHA1: Fecha de una entidad.
		\item ENTIDAD: Entidad que requiere de una fecha.
		\item CONDICION: La condición de relación que deben satisfacer FECHA1 Y FECHA2.
		\item FECHA2: Fecha de referencia.
		%\item ENTIDAD2: Entidad que requiere de una fecha.
	\end{itemize}
	\item[Ejemplo:] La fecha de autorización de un plan de estudios debe ser igual o anterior a la fecha actual.
	\item[Referenciado por: ] 
\end{mensaje}


%===========================  MSG50 ==================================
\begin{mensaje}{MSG50}{Correo de confirmación para la solicitud de dictamen}{Confirmación}
	\item[Canal:] Sistema
	\item[Propósito:] Notificar al actor que la relación entre una fecha ingresada y una fecha de referencia no es la esperada.
	\item[Redacción:] La $<FECHA1>$ de $<ENTIDAD>$ debe ser $<CONDICION>$ $<FECHA2>$
	\item[Parámetros:] 
	\begin{itemize}
		\item FECHA1: Fecha de una entidad.
		\item ENTIDAD: Entidad que requiere de una fecha.
		\item CONDICION: La condición de relación que deben satisfacer FECHA1 Y FECHA2.
		\item FECHA2: Fecha de referencia.
		%\item ENTIDAD2: Entidad que requiere de una fecha.
	\end{itemize}
	\item[Ejemplo:] La fecha de autorización de un plan de estudios debe ser igual o anterior a la fecha actual.
	\item[Referenciado por: ] 
\end{mensaje}

%===========================  MSG51 ==================================
\begin{mensaje}{MSG51}{Mensaje de notificación de calidad de alumno}{Notificación}
	\item[Canal:] Sistema
	\item[Propósito:] Notificar al actor que la relación entre una fecha ingresada y una fecha de referencia no es la esperada.
	\item[Redacción:] La $<FECHA1>$ de $<ENTIDAD>$ debe ser $<CONDICION>$ $<FECHA2>$
	\item[Parámetros:] 
	\begin{itemize}
		\item FECHA1: Fecha de una entidad.
		\item ENTIDAD: Entidad que requiere de una fecha.
		\item CONDICION: La condición de relación que deben satisfacer FECHA1 Y FECHA2.
		\item FECHA2: Fecha de referencia.
		%\item ENTIDAD2: Entidad que requiere de una fecha.
	\end{itemize}
	\item[Ejemplo:] La fecha de autorización de un plan de estudios debe ser igual o anterior a la fecha actual.
	\item[Referenciado por: ] 
\end{mensaje}

%===========================  MSG52 ==================================
\begin{mensaje}{MSG52}{Mensaje de notificación de términos y condiciones de la solicitud del dictamen}{Notificación}
	\item[Canal:] Sistema
	\item[Propósito:] Notificar al actor que la relación entre una fecha ingresada y una fecha de referencia no es la esperada.
	\item[Redacción:] La $<FECHA1>$ de $<ENTIDAD>$ debe ser $<CONDICION>$ $<FECHA2>$
	\item[Parámetros:] 
	\begin{itemize}
		\item FECHA1: Fecha de una entidad.
		\item ENTIDAD: Entidad que requiere de una fecha.
		\item CONDICION: La condición de relación que deben satisfacer FECHA1 Y FECHA2.
		\item FECHA2: Fecha de referencia.
		%\item ENTIDAD2: Entidad que requiere de una fecha.
	\end{itemize}
	\item[Ejemplo:] La fecha de autorización de un plan de estudios debe ser igual o anterior a la fecha actual.
	\item[Referenciado por: ] 
\end{mensaje}


%===========================  MSG53 ==================================
\begin{mensaje}{MSG53}{Responsabilidad de comprensión de dictamen}{Notificación}
	\item[Canal:] Sistema
	\item[Propósito:] Notificar al alumno la responsabilidad del resultado del dictamen.
	\item[Redacción:] Dictamen SESIÓN  Es tu responsabilidad acudir a Gestion Escolar para resolver tus dudas y obtener aclaraciones en caso de no haber entendido alguna sección de la respuesta de dictamen.
	\item[Parámetros:] 
	\begin{itemize}
		\item SESIÓN: La sesión celebrada en la cual fue emitida la respuesta de la solicitud de dictamen del alumno.
	\end{itemize}
	\item[Ejemplo:]	Dictamen 104-24-4-toSECEXT  Es tu responsabilidad acudir a Gestion Escolar para resolver tus dudas y obtener aclaraciones en caso de no haber entendido alguna sección de la respuesta de dictamen.
	\item[Referenciado por: ] 
\end{mensaje}

%===========================  MSG54 ==================================
\begin{mensaje}{MSG54}{Expiración de Token de confirmación}{Error}
	\item[Canal:] Sistema
	\item[Propósito:] Notificar al actor que el token de su solicitud de dictamen a expirado.
	\item[Redacción:] El token para confirmar tu solicitud de dictamen ha expirado.
	\item[Referenciado por: ] 
\end{mensaje}

%===========================  MSG55 ==================================
\begin{mensaje}{MSG55}{Solicitudes de dictámenes sin atender}{Confirmación}
	\item[Canal:] Sistema
	\item[Propósito:] Preguntar al actor si esta seguro que la información registrada es correcta.
	\item[Redacción:] ¿La información asentada es correcta y consistente?
	\item[Referenciado por: ] 
\end{mensaje}

%===========================  MSG56 ==================================
\begin{mensaje}{MSG56}{Dictaménes sin resolver}{Notificación}
	\item[Canal:] Sistema
	\item[Propósito:] Notificar al actor la existencia de solicitudes de dictámenes por atender.
	\item[Redacción:] No puedes finalizar la sesión de consejo, Faltan solicitudes de dictamen por dictaminar
	\item[Referenciado por: ] 
\end{mensaje}

\begin{mensaje}{MSG57}{Formato de archivo}{Notificación}
	\item[Canal:] Sistema
	\item[Propósito:] Notificar al actor que el tamaño máximo de un archivo debe ser de 5MB en formato PDF.
	\item[Redacción:] El archivo seleccionado no es válido. Éste debe tener un tamaño máximo de 5MB y en formato PDF.
	\item[Referenciado por: ] 
\end{mensaje}

%===========================  MSG58 ==================================
\begin{mensaje}{MSG58}{Operación fuera de periodo}{Error}
	\item[Canal:] Sistema
	\item[Propósito:] Notificar al actor que la operación que desea realizar está fuera del periodo permitido
	\item[Redacción:] No es posible realizar $<ARTICULO\_OPERACION>$ debido al que la fecha actual está fuera del periodo permitido para realizarla.
	\item[Parámetros:] 
	\begin{itemize}
		\item ARTICULO\_OPERACION: La operación que el actor desea llevar a cabo.
	\end{itemize}
	\item[Ejemplo:] No es posible realizar el envío de la solicitud debido a que la fecha actual está fuera del periodo permitido para realizarla.
	\item[Referenciado por: ] \refIdElem{HR-PR-CU1}
\end{mensaje}


%%TODO: ¿Algoritmo de qué?
%%TODO: No está el caso de uso al que hace referencia
%%===========================  MSG59 ==================================
%\begin{mensaje}{MSG59}{Se encontraron propuestas}{Notificación}
%	\item[Canal:] Sistema
%	\item[Propósito:] Notificar al actor que al ejecutar el algoritmo se encontraron combinaciones disponibles una vez que se terminó la asignacion de horarios a los grupos. 
%	\item[Redacción:] Se han encontrado $<N>$ combinaciones disponibles.
%	\item[Parámetros:] 
%	\begin{itemize}
%		\item N: El número de combinaciones disponibles que se han encontrado después de ejecutar el algoritmo.
%	\end{itemize}
%	\item[Ejemplo:] Se han encontrado 15 combinaciones disponibles.
%	\item[Referenciado por: ] \refIdElem{HR-UA-CU5}
%\end{mensaje}


%%TODO: ¿No es operación fallida?
%%TODO: No está agregado el caso de uso al que hace referencia
%%===========================  MSG60 ==================================
%\begin{mensaje}{MSG59}{No se pudieron crear los grupos}{Error}
%	\item[Canal:] Sistema
%	\item[Propósito:] Notificar al actor que no se pudieron generar los grupos para la asignación de turno.
%	\item[Redacción:] No se han podido generar los grupos para la asignación de turnos.
%	\item[Referenciado por: ] \refIdElem{HR-UA-CU5.2}
%\end{mensaje}


%===========================  MSG61 ==================================
\begin{mensaje}{MSG61}{Profesor asignado a un horario}{Error}
	\item[Canal:] Sistema
	\item[Propósito:] Notificar al actor que el profesor que desea asignar en ese horario ya se encuentra registrado en la misma u otra unidad académica y por lo cuál no se puede volver a asignar.
	\item[Redacción:] El profesor $<NOMBRE>$ ya se encuentra asignado en el horario $<HORARIO>$ en $<UNIDAD\_ACADEMICA>$.
	\item[Parámetros:] 
	\begin{itemize}
		\item NOMBRE: El nombre del profesor que ya se encuentra asignado.
		\item HORARIO: El horario en el que el profesor ya se encuentra asignado.
		\item UNIDAD\_ACADEMICA: La unidad académica en la que se encuentra asignado el profesor en el HORARIO.
	\end{itemize}
	\item[Ejemplo:] El profesor Jaime López Rabadán ya se encuentra asignado en el horario 10:30 - 12:00 en ESCOM.
	\item[Referenciado por: ] \refIdElem{HR-UA-GG-CU1.3}
\end{mensaje}


%===========================  MSG62 ==================================
\begin{mensaje}{MSG62}{Porcentaje de calificación erróneo}{Error}
	\item[Canal:] Sistema
	\item[Propósito:] Notificar al actor que el porcentaje de calificación asignado no es correcto.
	\item[Redacción:] La suma de los porcentajes asginados para la unidad de aprendizaje debe ser 100\%. Favor de revisar los datos introducidos.	
	\item[Referenciado por: ] \refIdElem{HR-UA-GG-CU1.3}
\end{mensaje}

%===========================  MSG63 ==================================
\begin{mensaje}{MSG63}{Actividad fuera de periodo}{Error}
	\item[Canal:] Sistema
	\item[Propósito:] Notificar al actor que la actividad definida se encuentra fuera del periodo establecido para la misma.
	\item[Redacción:] Las fechas de la actividad $<ACTIVIDAD>$ deben estar dentro del periodo $<PERIODO>$.
	\item[Parámetros:] 
	\begin{itemize}
		\item ACTIVIDAD: La actividad que se está fuera del periodo.
		\item PERIODO: El periodo dentro del cual debe estar definida la actividad.
	\end{itemize}
	\item[Ejemplo:] Las fechas de la actividad 'Envío de solicitudes' deben estar dentro del periodo escolar.
	\item[Referenciado por: ] \refIdElem{AP-DES-CU1.1.1}, \refIdElem{AP-DES-CU1.1.2}
\end{mensaje}

%===========================  MSG64 ==================================
\begin{mensaje}{MSG64}{Traslape de horarios}{Error}
	\item[Canal:] Sistema
	\item[Propósito:] Notificar al actor que el horario ya se encuentra definido anteriormente y se está generando un traslape.
	\item[Redacción:] Existe un traslape el día $<DIA>$ en el horario $<HORARIO>$ de las unidades de aprendizaje $<UNIDAD\_DE\_APRENDIZAJE1>$ y $<UNIDAD\_DE\_APRENDIZAJE2>$
	\item[Parámetros:] \cdtEmpty
	\begin{itemize}
		\item DIA: El día de la semana en el que se generó el traslape
		\item HORARIO: El horario en donde se generó el traslape.
		\item UNIDAD\_DE\_APRENDIZAJE1, UNIDAD\_DE\_APRENDIZAJE2: Las unidades de aprendizaje que tienen el traslape.
	\end{itemize}
	\item[Ejemplo:] Existe un traslape en el horario 10:30-12:00.
	\item[Referenciado por: ] \refIdElem{HR-UA-GG-CU1.2}
\end{mensaje}

%===========================  MSG65 ==================================
\begin{mensaje}{MSG65}{Horario no correspondiente}{Error}
	\item[Canal:] Sistema
	\item[Propósito:] Notificar al actor que no se cumple con la cantidad de horas semanales establecidas para la unidad de aprendizaje.
	\item[Redacción:] El número de horas que debe tener la unidad de aprendizaje $<UNIDAD\_APRENDIZAJE>$ es $<NUMERO\_HORAS>$ por semana. Favor de revisar los datos introducidos.
	\item[Parámetros:] \cdtEmpty
	\begin{itemize}
		\item UNIDAD\_APRENDIZAJE: La unidad de aprendizaje de la que no se cumplen con las horas establecidas.
		\item NUMERO\_HORAS: El número de horas establecidas para la unidad de aprendizaje.
	\end{itemize}
	\item[Ejemplo:] El número de horas que debe tener la unidad de aprendizaje Análisis Vectorial es 4.5 por semana. Favor de revisar los datos introducidos.
	\item[Referenciado por: ] \refIdElem{HR-UA-GG-CU1.2}
\end{mensaje}


%%TODO: Mejorar redacción
%%===========================  MSG66 ==================================
%\begin{mensaje}{MSG66}{Hora de inicio mayor que hora de inicio}{Error}
%	\item[Canal:] Sistema
%	\item[Propósito:] Noticar al actor que la hora de inicio es mayor que la hora de fin del turno.
%	\item[Redacción:] La hora de incio del horario es mayor que la hora del de fin del turno.
%	\item[Referenciado por: ] \refIdElem{HR-UA-GG-CU1.7}
%\end{mensaje}

%===========================  MSG67 ==================================
\begin{mensaje}{MSG67}{Hora inicial del turno vespertino}{Error}
	\item[Canal:] Sistema
	\item[Propósito:] Notificar al actor que la hora de inicio del turno vespertino debe ser la misma hora fin del turno matutitno
	\item[Redacción:] Se sugire que la hora de incio del turno vespertino sea la misma que la hora de fin del turno matutino.
	\item[Referenciado por: ] \refIdElem{HR-UA-GG-CU1.8}
\end{mensaje}