%=============================================
% Descripción de actores
\label{chapter:ActoresDelSistema}

	En el presente capítulo se definen los participantes en los procesos actuales y fortalecidos. Se da una breve descripción de cada uno, la mayoría tomada de la normatividad correspondiente: Reglamento General, Reglamento interno y Manuales de operación del IPN, entre otros. Aquellas descripciones tomadas de la normatividad tienen la referencia del documento de dónde fueron tomadas, en algunos casos los actores son propuestos para poder llevar a cabo algunas de las tareas en el \refElem{Calmecac}.

\section{Descripción de Actores}

Cada participante se describe con la siguiente estructura:

\begin{objetivos}[Descripción de participantes]
	\item {\bf Nombre:} Nombre con el que se le conoce al participante dentro del negocio.
	\item {\bf Descripción:} Descripción breve del perfil, puesto o rol que juega el participante.
	\item {\bf Área:} En caso de que el participante pertenezca a una estructura se indica el área, departamento o ubicación al que pertenece o se encuentra.
	\item {\bf Responsabilidades:} Se listan las responsabilidades oficiales o relacionadas con el Calmécac según aplique. En el caso de las descripciones tomadas de la normatividad se listan todas.
	\item {\bf Perfil:} Cuando se reconoce un nivel de conocimientos, preparación o capacidades determinadas para desarrollar una participación se describen, con la finalidad de ser tomado en cuenta para el desarrollo del sistema.
	\item {\bf Fuente:} Referencia al documento base de donde fue tomada dicha descripción.
\end{objetivos}

Cuando el participante es un departamento o dirección, se omite el dato ``Perfil'' es un Sistema de Información los datos que se describen son:

\begin{objetivos}[Descripción de sistemas]
	\item {\bf Nombre:} Nombre o siglas del sistema.
	\item {\bf Descripción:} Descripción breve del sistema.
	\item {\bf Área:} Área al que le pertenece, opera o es responsable del sistema.
	\item {\bf Responsabilidades:} Funciones principales del sistema.
	\item {\bf Lenguaje:} Tecnologías principales o plataforma utilizadas en su desarrollo.
	\item {\bf Fuente:} Referencia al documento base de donde fue tomada dicha descripción.
\end{objetivos}

\section{Participantes detectados}

%==========UAResponsableEstructuraEducativa ======%
\begin{actor}{UAResponsableEstructuraEducativa}{Reponsable de Estructura Educativa en la Unidad Académica}{Persona encarga de}

	\item[Área:] \refElem{UnidadAcademica}
	\item[Responsabilidades:] \cdtEmpty
	\begin{itemize}
		
		
		\item Definir la oferta educativa para los diferentes programas académicos que hay en la unidad académica en sus distintas modalidades para el siguiente periodo escolar.
		\item Definir la cantidad de grupos a utilizar para la oferta educativa de la unidad académica para el siguiente periodo escolar.
		\item Asignar unidades de aprendizaje a los grupos que se utilizaran para le siguiente periodo escolar.
		\item Definir los horarios de las unidades de aprendizaje asignadas a los grupos.
		\item Definir más de un laboratorio a las unidades de aprendizaje que lo requieran.
		\item Asignar profesores a las unidades de aprendizaje que se ofertarán para el próximo periodo escolar.
		\item Asignar a todos los profesores basificados su carga máxima de horas frente a grupo.
		\item Asignar a los profesores interinos las unidades de aprendizaje 
		\item Definir un periodo para la recepción de las solicitudes de aprendizaje que son enviadas por los profesores.
		
	\end{itemize}

	%fuente
	\item[Fuente:]%Reglamento General de Estudios, Manual de Organización de la Secretaría Académica
	%\item[Solicitado por:] 


\end{actor}


%========== Responsable de Horarios en la UA ======%
\begin{actor}{UAResponsableHorarios}{Responsable de Horarios en la Unidad Académica}{Persona encarga de}
	
	\item[Área:] \refElem{UnidadAcademica}
	\item[Responsabilidades:] \cdtEmpty
	\begin{itemize}
		\item Monitorear las solicitudes de unidades de aprendizaje de los profesores
		\item Enviar una notificación a los profesores que no halla enviado aún su solicitud de unidades de aprendizaje antes de que se acabe el periodo para la recepción de solicitudes.
		\item Consultar las solicitudes de los profesores para las diferentes academias que existen en la unidad académica.
	\end{itemize}

	%fuente
	\item[Fuente:]%Reglamento General de Estudios, Manual de Organización de la Secretaría Académica
	%\item[Solicitado por:] 
	
\end{actor}

%==========Auxiliar de Soporte Documental en la UA ======%
\begin{actor}{UAAuxiliarSD}{Auxiliar de Soporte Documental de la Unidad Académica}{Persona encarga de}
	
	\item[Área:] 
	\item[Responsabilidades:] \cdtEmpty
	\begin{itemize}
		\item 
		\item 
	\end{itemize}
	
	%fuente
	\item[Fuente:] 
	
	%\item[Solicitado por:] 
	
\end{actor}

%==========Analista de la DES ======%
\begin{actor}{DESAnalista}{Analista de la Dirección de Educación Superior}{Persona encarga de}
	
	\item[Área:] 
	\item[Responsabilidades:] \cdtEmpty
	\begin{itemize}
		\item 
		\item 
	\end{itemize}
	
	%fuente
	\item[Fuente:] 
	
	%\item[Solicitado por:] 
	
\end{actor}

%========== Responsable Estructura Educativa DES ======%
\begin{actor}{DESResponsableEE}{Responsable de Estructura Educativa de la Dirección de Educación Superior}{Persona encarga de}
	
	\item[Área:] 
	\item[Responsabilidades:] \cdtEmpty
	\begin{itemize}
		\item 
		\item 
	\end{itemize}
	
	%fuente
	\item[Fuente:] 
	
	%\item[Solicitado por:] 
	
\end{actor}

%================Profesor===================%
\begin{actor}{Profesor}{Profesor}{Persona que conduce el proceso de enseñanza-aprendizaje enfocado a la transmisión de conocimientos y a la formación integral del alumno.será la autoridad académica del grupo a su cargo y desempeñará sus actividades conforme al principio de libertad de cátedra e investigación, atendiendo los programas aprobados por las autoridades académico-administrativas del IPN y del centro de trabajo correspondiente.}
	
	
	%area
	\item[Área:] Unidad Académica
	
	%responsabilidades
	\item[Responsabilidades:]\cdtEmpty
	\begin{itemize}
		\item Tiene la obligación de profesar cátedra según el número de
		horas que ampare su nombramiento y de acuerdo con la distribución de las actividades que hagan las
		autoridades correspondientes, conforme a las funciones específicas de su categoría académica.
		\item Deberá cumplir con los programas de su asignatura aprobados
		por las autoridades académicas correspondientes. Cuando por causas no imputables al personal
		académico, no se cumplan dichos programas, se convendrá la satisfacción de los mismos con las
		autoridades del centro de trabajo.
		\item Deberá realizar exámenes en su centro de trabajo y dentro del
		período que indica el calendario escolar, así como entregar resultados de los mismos, de acuerdo a los establecido en el Reglamento Interno del I.P.N.
		\item Otorgará el crédito necesario al centro de adscripción en los trabajos publicados y que se hayan realizado dentro del mismo.
	\end{itemize}
	
	%perfil
	\item[Pérfil:]\cdtEmpty
	\begin{itemize}
		\item Licenciatura como mínimo, o grado de Maestría/Doctorado
		\item Haber tenido un nombramiento
		\item Haber presentado el examen de oposición con un puntaje mínimo de 80
	\end{itemize}
	
	%fuente
	\item[Fuente:]Reglamento de las Condiciones Interiores de Trabajo del Personal Académico del I.P.N.
	
	%\item[Solicitado por:] Chadwick
\end{actor}

%================UAResponsableSD===================%
\begin{actor}{UAResponsableSD}{Responsable de Soporte Documental de la Unidad Académica}{Persona que conduce el proceso de enseñanza-aprendizaje enfocado a la transmisión de conocimientos y a la formación integral del alumno.será la autoridad académica del grupo a su cargo y desempeñará sus actividades conforme al principio de libertad de cátedra e investigación, atendiendo los programas aprobados por las autoridades académico-administrativas del IPN y del centro de trabajo correspondiente.}
	
	
	%area
	\item[Área:] Unidad Académica
	
	%responsabilidades
	\item[Responsabilidades:]\cdtEmpty
	\begin{itemize}
		\item Tiene la obligación de profesar cátedra según el número de
		horas que ampare su nombramiento y de acuerdo con la distribución de las actividades que hagan las
		autoridades correspondientes, conforme a las funciones específicas de su categoría académica.
		\item Deberá cumplir con los programas de su asignatura aprobados
		por las autoridades académicas correspondientes. Cuando por causas no imputables al personal
		académico, no se cumplan dichos programas, se convendrá la satisfacción de los mismos con las
		autoridades del centro de trabajo.
		\item Deberá realizar exámenes en su centro de trabajo y dentro del
		período que indica el calendario escolar, así como entregar resultados de los mismos, de acuerdo a los establecido en el Reglamento Interno del I.P.N.
		\item Otorgará el crédito necesario al centro de adscripción en los trabajos publicados y que se hayan realizado dentro del mismo.
	\end{itemize}
	
	%perfil
	\item[Pérfil:]\cdtEmpty
	\begin{itemize}
		\item Licenciatura como mínimo, o grado de Maestría/Doctorado
		\item Haber tenido un nombramiento
		\item Haber presentado el examen de oposición con un puntaje mínimo de 80
	\end{itemize}
	
	%fuente
	\item[Fuente:]Reglamento de las Condiciones Interiores de Trabajo del Personal Académico del I.P.N.
	
	%\item[Solicitado por:] Chadwick
\end{actor}
