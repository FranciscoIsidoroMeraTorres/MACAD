% !TEX root = ../integrado.tex

\section{Propósito del Documento}

Este documento es el primer entregable del proyecto MACAD el cual está compuesto por la descripción de procesos TO-BE utilizando la notación B.P.M.N. en su segunda versión cuya especificación está con base en la notación U.M.L. por el O.M.G. Así mismo este documento contiene la especificación de los requerimientos iniciales del sistema asociados con un criterio que permitirá evaluar la satisfacción proporcionada por la implementación y puesta en producción del sistema \varSistema. El documento tiene como propósito establecer los acuerdos necesarios para la implementación del sistema. 


\section{Estructura}

El documento está compuesto por los siguientes capítulos:

\begin{itemize}
	\item El capítulo \ref{ch:analisis} contiene un análisis de la F.A.M.A. y su situación actual así como su visión y misión modificada con base en los servicios que puede llegar a brindar no sólo a estudiantes sino a artistas o empresas dedicadas al entretinimiento.
	
	\item En el capítulo \ref{ch:glosario} se encontrará un glosario de términos utilizados comúnmente en el negocio así como la incorporación de nuevos elementos que permitan alcanzar la visión y misión de la academia.
	
	\item En el capítulo \ref{ch:procesos} se encontrará la definición de los procesos que se modificarán para lograr alcanzar la visión de F.A.M.A. en un tiempo estimado de TANTOS AÑOS implementando un nuevo un modelo de negocio basado en suscripciones y de cola larga.
	
	\item En el capítulo \ref{ch:reqFunc} se encontrarán los requerimientos obtenidos a raíz de las juntas con el cliente y con base en el modelo de negocio que se utilizarán para implementar el sistema.
	
	\item En el capítulo \ref{ch:reqNoFunc} se encontrarán los requerimientos obtenidos a raíz de las limitaciones y restricciones en las que se deberá implementar el sistema MACAD.
	
	\item En el capítulo \ref{ch:criterio} se encontrarán los acuerdos para verificar que los requerimientos funcionales hayan sido cubiertos con el sistema, en su mayoría de forma satisfactoria.
\end{itemize}


\section{Nomenclatura}

Para apoyar a la lectura de este documento en esta sección se presentan la diferente nomenclatura utilizada en el documento para identificar : procesos, requerimientos y criterios de evaluación como se muestra en la siguiente tabla.


\begin{table}[hbtp!]
	\begin{center}
		\begin{tabular}{|p{0.7\textwidth}|}
			\hline
			\rowcolor{colorAgua}
			\begin{description}
				\item[PRXX] Indicador de un proceso principal del cual se derivan otros. Las XX indican un número entero que puede ir del \textit{00} hasta \textit{99} 
				\item[SPRXX] Indicador de un subproceso en donde se especifican las tareas que deben llevar a cabo 
				
				\item[RF-MM-XX] Indicador de los requerimientos funcionales. Las MM indican el módulo al que se asignó el requerimiento de acuerdo a a la necesidad que se va a cubrir. Las XX indican un número entero que puede ir del \textit{00} hasta \textit{99}.
				
				\item[RNF-XX] Indicador de los requerimientos no funcionales del sistema. Debido a que estos requerimientos tienden a restringir el desarrollo del sistema no tienen un módulo asociado.
				
				\item[CR-MM-XX] Indicador de criterios de evaluación para los requerimientos funcionales. Al igual que los requerimientos funcionales estos contienen la nomenclatura MM para indicar el módulo al que pertenece el criterio.
			\end{description}\\
			\hline
		\end{tabular}
	\end{center}
\end{table}



