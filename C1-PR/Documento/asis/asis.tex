% !TEX root = ../integrado.tex



\section{Introducción}
En este capítulo se presenta el análisis correspondiente de los procesos AS-IS de F.A.M.A. Un proceso AS-IS es la descripción de las acciones o tareas llevadas a cabo actualmente por los diferentes participantes involucrados en el proceso de negocio. Esta descripción tiene como propósito apoyar al descubrimiento de problemáticas que impiden el crecimiento de la empresa así como la identificación de áreas de oportunidad que pudieran llegar a ser implementadas en el negocio parea mejorar los servicios o productos que se ofertan.


\section{Notación}

Los procesos de negocio en este capítulo son los más relevantes de F.A.M.A. y los cuales se toman como referencia para el análisis, diseño e implementación de un sistema que apoye en la resolución de las problemáticas a las que la academia se enfrenta. Cada proceso tiene su sección correspondiente presentando un resumen general del proceso así como su diagrama correspondiente. Los diagramas están basados en la notación especificada por la O.M.G. del Business Process Modeling Notation en su versión 2.0. Posterior a su resumen se expone una tabla de elementos que permiten la ejecución del proceso. Finalmente se enlista cada tarea o subproceso realizado por los participantes hasta la conclusión del proceso.

% !TEX root = ../integrado.tex


\begin{Proceso}{PR01}{Ejecución de Trimestre}{La academia F.A.M.A. organiza sus actividades en periodos de tres meses. En un inicio el \refElem{Director} notifica a los alumnos de la escuela un fecha propuesta para el inicio de actividades. Una vez llegada la fecha los alumnos se presentan a la academia en donde el Director solicita los horarios de disponibilidad de cada uno y selecciona los cursos que se impartirán en el trimestre. En todo momento durante la ejecución del trimestre el \refElem{Administrador} busca a los alumnos para solicitar el pago correspondiente a su inscripción, ya sea por los cursos que los alumnos toman o por su carrera inscrita. Mientras se ejecuta el trimestre, el \refElem{Director} busca patrocinios de diferentes empresas y también solicita licencias a su proveedor \textit{Yamaha México} }{\Pfig[1]{asis/images/PR01-TrimestreASIS}{PR01}{Ejecución del Trimestre}}
	
	\PRitem{Participantes}{\begin{Titemize}
		\Titem \refElem{Director}
		\Titem \refElem{Administrador}
		\Titem \refElem{Profesor}
		\Titem \refElem{Alumno}
		\Titem \refElem{Proveedor}
	\end{Titemize}}	
	\PRitem{Entradas}{\begin{Titemize}
			\Titem Cursos(\refElem{tCurso}).
			\Titem \refElem{tHorarioDeDisponibilidad}.
			\Titem \refElem{tLicencia}.
			\Titem \refElem{tPago}.
			\Titem \refElem{tAsistencia}.
		\end{Titemize}}
	\PRitem{Salidas}{\begin{Titemize}
		\Titem \refElem{tCertificado}.
		\Titem \refElem{tApuntes}.
		\Titem \refElem{tEvaluacion}. 
	\end{Titemize}}

\end{Proceso}


\begin{PDescripcion}
	\Ppaso[\EinicioTiempo] El proceso inicia cada tres meses. El \refElem{Director} notifica a los alumnos del inicio de un nuevo trimestre.
	\Ppaso[\PSubProceso]\cdtLabelTask{SPR01}{Especificar fechas de nuevo trimestre} El \refElem{Director} planifica el inicio y fin de un nuevo trimestre. Continúa con la tarea \cdtRefTask{SPR02}{Dar inicio a trimestre}.
	
	\Ppaso[\PSubProceso]\cdtLabelTask{SPR02}{Dar inicio a trimestre} El \refElem{Director} cita a los alumnos a asistir el día de inicio de trimestre.
	
	\Ppaso[\iCompuerta] Una vez iniciado el trimestre se realizarán paralelamente las actividades \cdtRefTask{SPR03}{Cobrar a alumnos}, \cdtRefTask{SPR04}{Definir cursos y horarios}, \cdtRefTask{SPR07}{Buscar patrocinios} y \cdtRefTask{SPR08}{Solicitar Licencia}
	
	\Ppaso[\PSubProceso] \cdtLabelTask{SPR03}{Cobrar a alumnos} El \refElem{Administrador} a lo largo del trimestre solicitará el cobro por los cursos inscritos por el alumno en el trimestre.
	
	\Ppaso[\PSubProceso] \cdtLabelTask{SPR04}{Definir cursos y horarios} El \refElem{Director} durante una clase solicitará a los alumnos su \refElem{tHorarioDeDisponibilidad} para poder planificar el horario semanal de los cursos. Así mismo seleccionará los cursos que se ofertarán aleatoriamente o siguiendo la secuencia de los cursos ofertados el trimestre pasado. Continua con el sub proceso \cdtRefTask{SPR05}{Iniciar clases}.
	
	\Ppaso[\PSubProceso] \cdtLabelTask{SPR07}{Buscar patrocinios} El \refElem{Director} a lo largo del trimestre y de forma paralela buscará obtener patrocinios de diferentes empresas que puedan otorgar licencias estudiantiles o precios especiales de productos que puedan ser otorgadas a los alumnos. Esta tarea no tiene continuación.
	
	\Ppaso[\PSubProceso] \cdtLabelTask{SPR08}{Solicitar Licencia} El \refElem{Director} a lo largo del trimestre solicitará a su \refElem{Proveedor} un conjunto de licencias de software para los estudiantes. Está tarea no tiene continuación.
	
	\Ppaso[\PSubProceso] \cdtLabelTask{SPR05}{Iniciar Clases} El \refElem{Director} marca el inicio de las clases.
	
	\Ppaso[\iCompuerta] Una vez iniciadas las clases se llevarán a cabo las siguientes tareas en paralelo: \cdtRefTask{SPR06}{Dar Clase}, \cdtRefTask{SPR09}{Vigilar desempeño de profesores}.
	
	\Ppaso[\PSubProceso] \cdtLabelTask{SPR06}{Dar Clase} El \refElem{Profesor} le proporcionará apuntes a los alumnos inscritos en su curso por medio de la plataforma de Moodle o por medios escritos. Este evento sucederá hasta que los temas hayan sido cubiertos en su totalidad o al 80\%. Concluida este subproceso se continua con \cdtRefTask{SPR07}{Evaluar Alumnos}.
	
	\Ppaso[\PSubProceso] \cdtLabelTask{SPR07}{Evaluar Alumnos} El \refElem{Profesor} evalúa a los alumnos por medio de un examen escrito o utilizando la plataforma Moodle. El \refElem{Alumno} tiene derecho a presentar el examen dos veces.
	
	\Ppaso[\PSubProceso] \cdtLabelTask{SPR09}{Vigilar desempeño de profesores} El \refElem{Director} en conjunto con el \refElem{Administrador} desarrollan una prueba que los alumnos contestan para calificar el desempeño de los profesores.
	
	\Ppaso[\iCompuerta] Para continuar el flujo los subprocesos \cdtRefTask{SPR07}{Evaluar alumnos} y \cdtRefTask{SPR09}{Vigilar desempeño de profesores} deben terminar. El siguiente proceso a llevar a cabo es el \cdtRefTask{SPR10}{Generar Certificados}.
	
	\Ppaso[\PSubProceso]\cdtLabelTask{SPR10}{Generar Certificados} El 
	
\end{PDescripcion}
