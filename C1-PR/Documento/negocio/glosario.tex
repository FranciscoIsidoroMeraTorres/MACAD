% !TEX root = ../integrado.tex

En este capítulo se encuentran los diferentes términos de negocio comúnmente utilizados en la academia F.A.M.A. por el \refElem{Director}, \refElem{Administrador} y \refElem{Profesor} para comunicar ideas o plantear cambios o sugerencias a los cursos o servicios.

\section{Glosario de términos}

\begin{bGlosario}
	
	\bTerm[Alumnos]{tAlumno}{Alumno}{Persona que paga por tomar uno o varios cursos, o una carrera que F.A.M.A. ofrece.}
	
	\bTerm{tApuntes}{Apuntes}{Se conoce como apuntes a los contenidos de los diferentes temas que conforman a un curso.}
	
	\bTerm{tAsistencia}{Asistencia}{Es la acción que un Alumno realiza cuando toma su clase en la hora y fecha indicada para tal propósito.}
	
	\bTerm{tBandaMusical}{Banda Musical}{Grupo de personas que desean que se les grabe, edite o que produzca material discográfico digital o físico.}
	
	\bTerm{tCarrera}{Carrera}{Conjunto de cursos que tienen como propósito formar a profesionales aptos en una rama específica del conocimiento.}
	
	\bTerm{tCertificado}{Certificado}{Documento expedido cuando un alumno a concluido un curso con una calificación arriba del 80\%.}
	
	\bTerm{tCurso}{Curso}{Temas recopilados y planificados de forma secuencial para adquirir conocimiento sobre una materia en específico}
	
	\bTerm{tEmpresa}{Empresa}{Organización que paga por la capacitación de su personal en una o varias tecnologías en específico.}
	
	\bTerm{tEvaluacion}{Evaluación}{Al medio digital o escrito que pone a prueba los conocimientos adquiridos por un alumno durante un curso.}
	
	\bTerm{tHorarioDeDisponibilidad}{Horario de Disponibilidad}{Días y horas de la semana con las que un alumno cuenta para tomar clase en la academia de forma virtual o de forma presencial.}
	
	\bTerm{tLicencia}{Licencia}{Clave, llave o dispositivo que permite utilizar un determinado software o varios programas de un mismo desarrollador, por un periodo de tiempo limitado o ilimitado, esto con base en el convenio establecido con F.A.M.A.}
	
	\bTerm{tPago}{Pago}{Dinero que se da a la academia por un servicio, curso o carrera. }
	
	\bTerm{tProfesor}{Profesor}{Persona que domina un conjunto de conocimientos los cuales transmite a un grupo de personas.}

	\bTerm{tServicioDeCapacitacion}{Servicio de Capacitación}{Conjunto de acciones que se le brinda a una empresa para que su personal se actualice en el uso y manipulación de un conjunto de tecnologías en específico.}
	
	\bTerm{tServiciodeGrabacion}{Servicio de Grabación}{Conjunto de acciones que se le brinda a un músico o banda musical que les permitan tener material discográfico digital o físico grabado, editado o producido en las instalaciones de la academia.}
	 		
	\bTerm{tStack}{Stack de Tecnologías}{Conjunto de cursos que tienen como propósito enseñar a un alumno el uso y manejo de una tecnología en específico.}

\end{bGlosario}
