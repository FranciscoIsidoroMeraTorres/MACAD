% No comentar las reglas cuyo estatus es APROBADO.
%---------------------------------------------------------
\section{Reglas de Negocio}

% Use la siguiente plantilla para crear una regla de negocio.
%%======================================================================
%\begin{BusinessRule}{SUB-BRXXX}{Nombre de la regla}
%	{elija Clase}    % Clase: \bcCondition,   \bcIntegridad, \bcAutorization, \bcDerivation.
%	{elija Tipo}     % Tipo:  \btEnabler,     \btTimer,      \btExecutive.
%	{elija Nivel}    % Nivel: \blControlling, \blInfluencing.
%	\BRItem[Versión] 1.0.
%	\BRItem[Estado] Propuesta.
%	\BRItem[Propuesta por] \TODO{Escriba su nombre.}
%	\BRItem[Revisada por] Pendiente.
%	\BRItem[Aprobada por] Pendiente.
%	\BRItem[Descripción] \TODO{Redacte la regla lo más claro posible.}
%	\BRItem[Sentencia] \TODO{Redacte la regla lo más formalmente posible, puede apoyarse de : pseudocódigo, algoritmo o notación matemática.}
%	\BRItem[Motivación] \TODO{Razón o causa de la regla. Puede describir lo que se desea evitar con la regla.}
%	\BRItem[Ejemplo positivo] Cumplen la regla:
%		\begin{itemize}
%			\item \TODO{Redacte preferentemente 3 ejemplos en los que la regla se cumple}
%		\end{itemize}
%	\BRItem[Ejemplo negativo] No cumplen con la regla:
%		\begin{itemize}
%			\item \TODO{Redacte preferentemente 3 ejemplos en los que la regla NO se cumple}
%		\end{itemize}
%	\BRItem[Referenciado por] \TODO{Referencíe los casos de uso en dónde se cita esta regla}
%\end{BusinessRule}

%=======================================
%==== REGLAS DE NEGOCIO DEL SISTEMA ====
%=======================================

\subsection{Reglas de Negocio del Sistema}

%%======================================================================
\begin{BusinessRule}{BR-S001}{Campos obligatorios}
	{\bcIntegridad}    % Clase: \bcCondition,   \bcIntegridad, \bcAutorization, \bcDerivation.
	{\btEnabler}     % Tipo:  \btEnabler,     \btTimer,      \btExecutive.
	{\blControlling}    % Nivel: \blControlling, \blInfluencing.
	\BRItem[Versión] 1.0.
	\BRItem[Estado] Propuesta.
	\BRItem[Propuesta por] Ángeles
	\BRItem[Revisada por] Pendiente.
	\BRItem[Aprobada por] Pendiente.
	\BRItem[Descripción] Los campos proporcionados al sistema marcados como obligatorios no se deben omitir.
	\BRItem[Sentencia] Sea $campo$ un atributo de $Entidad$, tal que $campo.obligatorio = true$ entonces
	$ \forall campo \Rightarrow campo.valor \neq \emptyset $
	
	\BRItem[Motivación] Evitar la falta de información relevante en el sistema causada por la omisión del actor al introducir datos.
	\BRItem[Ejemplo positivo] Cumplen la regla:
	\begin{itemize}
		\item Para la entidad unidad aprendizaje, el actor introduce todos los atributos solicitados.
		\item Para el Perfil Docente, el actor introduce todos los datos con excepción de habilidades y actitudes.
		\item Para la entidad bibliografía, el actor introduce todos los atributos solicitados con excepción del ISSN.
	\end{itemize}
	\BRItem[Ejemplo negativo] No cumplen con la regla:
	\begin{itemize}
		\item El actor no proporciona el nombre para la entidad
		\item El actor no proporciona la fecha de validación para la entidad \refElem{planEstudio}
		\item El actor no proporciona el lugar de realización para la entidad \refElem{practica}.
	\end{itemize}
	\BRItem[Referenciado por] 

\end{BusinessRule}


%%======================================================================
\begin{BusinessRule}{BR-S002}{Información correcta}
	{\bcIntegridad}    % Clase: \bcCondition,   \bcIntegridad, \bcAutorization, \bcDerivation.
	{\btEnabler}     % Tipo:  \btEnabler,     \btTimer,      \btExecutive.
	{\blControlling}    % Nivel: \blControlling, \blInfluencing.
	\BRItem[Versión] 1.0.
	\BRItem[Estado] Propuesta.
	\BRItem[Propuesta por] Ángeles
	\BRItem[Revisada por] Pendiente.
	\BRItem[Aprobada por] Pendiente.
	\BRItem[Descripción] Todos los datos proporcionados al sistema deben respetar su formto, estar dentro de su longitud máximao mínima descrita en el diccionario de datos y pertenecer al tipo de dato especificado en el modelo de información.
	\BRItem[Sentencia] Sea $formato$ la expresión regular que determina el formato de un campo definido en el diccionario de datos, $L$ el lenguaje que genera $formato$ y $campo$ un campo introducido por el actor, entonces $ \forall campo \Rightarrow campo \in L $.
	\BRItem[Sentencia] Sea $dato$ un elemento definido dentro de un tipo de dato $D$, entonces $ \forall dato \Rightarrow dato \in D $.
	\BRItem[Motivación] Mantener los datos del sistema dentro del formato definido y dentro del conjunto de datos correspondiente.
	\BRItem[Especificación de formatos]

	\begin{itemize}
		\item Periodo escolar: cadena1-cadena2/Número.
		\item El actor introduce un número de teléfono que contiene sólo números.
		\item El actor introduce un correo electrónico que contiene un símbolo '@' y cuya terminación es un dominio de correo electrónico.
	\end{itemize}


	\BRItem[Ejemplo positivo] Cumplen la regla:
	\begin{itemize}
		\item El actor introduce el nombre de una \refElem{unidadAcademica} que solamente contiene caractéres alfabéticos.
		\item El actor introduce un número de teléfono que contiene sólo números.
		\item El actor introduce un correo electrónico que contiene un símbolo '@' y cuya terminación es un dominio de correo electrónico.
	\end{itemize}
	\BRItem[Ejemplo negativo] No cumplen con la regla:
	\begin{itemize}
		\item El actor introduce un nombre de \refElem{unidadAcademica} que contiene el símbolo '@'.
		\item El actor introduce un número de teléfono que contiene símbolos alfabéticos.
		\item El actor introduce un correo electrónico que no contiene el carácter '@'.
	\end{itemize}

	\BRItem[Referenciado por] 

\end{BusinessRule}

%======================================================================
\begin{BusinessRule}{BR-S003}{Eliminación lógica de elementos}
	{\bcCondition}    % Clase: \bcCondition,   \bcIntegridad, \bcAutorization, \bcDerivation.
	{\btEnabler}     % Tipo:  \btEnabler,     \btTimer,      \btExecutive.
	{\blControlling}    % Nivel: \blControlling, \blInfluencing.
	\BRItem[Versión] 1.0.
	\BRItem[Estado] Propuesta.
	\BRItem[Propuesta por] Ángeles Cerritos
	\BRItem[Revisada por] Pendiente.
	\BRItem[Aprobada por] Pendiente.
	\BRItem[Descripción] Un elemento sólo se puede eliminar si no tiene asociaciones con otros elementos. Tomando en cuenta la definición de la sentencia, se tienen los siguietes valores para $e_1$ y $e_2$, respectivamente:
	
	\begin{itemize}
		\item Espacio y Edificio.
		\item Plan de Estudio y Programa Académico.
		\item Espacio y Nivel.
	\end{itemize}
	
	\BRItem[Sentencia] Sean $ e_1 \in Entidad1 , e_2 \in Entidad2, R(x,y) = x $ está asociado con $ y $ entonces
	
	$ e_2 $ se puede eliminar si y sólo si $ \nexists e_1 $ tal que $ R(e_1, e_2) $ se cumpla.
	
	
	\BRItem[Motivación] Evitar que existan elementos en el sistema asociados a elementos que ya no existen dentro de él.
	\BRItem[Ejemplo positivo] Cumplen la regla:
	\begin{itemize}
		\item Eliminar un edificio que no tiene espacios asociados.
		\item Eliminar un Programa Académico que no tiene planes de estudio asociados.
		\item Eliminar los espacios asociados a un edificio y después eliminar el edificio.
	\end{itemize}
	\BRItem[Ejemplo negativo] No cumplen con la regla:
	\begin{itemize}
		\item Eliminar un edificio con un espacio asociado.
		\item Eliminar un edificio con tres espacios asociados.
		\item Eliminar un Programa Académico con un Plan de Estudio asociado.
	\end{itemize}
	\BRItem[Referenciado por] 
\end{BusinessRule}

%%======================================================================
\begin{BusinessRule}{BR-S004}{Unicidad de elementos}
	{\bcIntegridad}    % Clase: \bcCondition,   \bcIntegridad, \bcAutorization, \bcDerivation.
	{\btEnabler}     % Tipo:  \btEnabler,     \btTimer,      \btExecutive.
	{\blControlling}    % Nivel: \blControlling, \blInfluencing.
	\BRItem[Versión] 1.0.
	\BRItem[Estado] Propuesta.
	\BRItem[Propuesta por] Ángeles.
	\BRItem[Revisada por] Pendiente.
	\BRItem[Aprobada por] Pendiente.
	\BRItem[Descripción] Un elemento no se puede duplicar en el ámbito donde es utilizado ni registrarse en más de una ocasión. Dada la sentencia se consideran dentro de la regla las siguientes entidades y atributos:
	
	\begin{itemize}
		\item Unidad Académica, \{nombre, acrónimo\}
		\item Programa Académico, \{nombre\}
		\item Unidad de Aprendizaje, \{nombre, Plan de Estudio, Programa Académico, Unidad Académica\}
		\item Bibliografía, \{ISBN\}
		\item Plan de Estudio, \{nombre, Programa Académico, Unidad Académica\}
		\item Edificio, {Nombre por Unidad Académica}
		\item Espacio, {Nombre por Edificio por Unidad Académica}

	\end{itemize}
	\BRItem[Sentencia] Sean $ e_1, e_2 \in Entidad , atributos = {atributo_1, atributo_2,...,atributo_n}$ atributos de la entidad tal que:\\
	Si $ \forall a \in atributos$ se cumple que $ e_1.a = e_2.a$ entonces $ e_1 = e_2 $.
	
	\BRItem[Motivación] Evitar la duplicidad de elementos dentro del sistema.
	\BRItem[Ejemplo positivo] Cumplen la regla:
	\begin{itemize}
		\item Dos Unidades de Aprendizaje con diferentes nombres.
		\item Dos prácticas con el mismo lugar de realización pero diferentes nombres.
		\item Dos Unidades Académicas con la misma dirección pero diferente nombre.
	\end{itemize}
	\BRItem[Ejemplo negativo] No cumplen con la regla:
	\begin{itemize}
		\item Dos Unidades de Aprendizaje con el mismo nombre.
		\item Dos Bibliografías con el mismo ISBN.
		\item Dos Planes de Estudio con el mismo nombre.
	\end{itemize}
	\BRItem[Referenciado por] 
\end{BusinessRule}

%%======== BR-S005 Eiminación de Bibliografía ===============


%=======================================
%==== REGLAS DE NEGOCIO DEL NEGOCIO ====
%=======================================

\subsection{Reglas de Negocio del Negocio}

%%======================================================================
\begin{BusinessRule}{BR-N025}{Periodo válido}
	{\bcCondition}    % Clase: \bcCondition,   \bcIntegridad, \bcAutorization, \bcDerivation.
	{\btEnabler}     % Tipo:  \btEnabler,     \btTimer,      \btExecutive.
	{\blControlling}    % Nivel: \blControlling, \blInfluencing.
	\BRItem[Versión] 1.0.
	\BRItem[Estado] Propuesta.
	\BRItem[Propuesta por] David
	\BRItem[Revisada por] Pendiente.
	\BRItem[Aprobada por] Pendiente.
	\BRItem[Descripción] Un periodo de solicitudes de unidades de aprendizaje es válido, sí y solo sí se encuentra dentro del periodo escolar actual.
%	\BRItem[Sentencia]  $ $
%	\BRItem[Motivación] Evitar que se defina un periodo de recepción de solicitudes de unidades de aprendizaje en un tiempo posterior a la planeación de la estructura educativa.
%	\BRItem[Ejemplo positivo] \cdtEmpty
%	\begin{itemize}
%		\item 
%		\item 
%		\item 
%	\end{itemize}
%	\BRItem[Ejemplo negativo]
%	\begin{itemize}
%		\item 
%		\item 
%		\item 
%	\end{itemize}
%	\BRItem[Referenciado por] \refIdElem{UA-CU2.1}
\end{BusinessRule}

%%======================================================================
\begin{BusinessRule}{BR-N026}{Relación entre fechas de periodo}
	{\bcCondition}    % Clase: \bcCondition,   \bcIntegridad, \bcAutorization, \bcDerivation.
	{\btEnabler}     % Tipo:  \btEnabler,     \btTimer,      \btExecutive.
	{\blControlling}    % Nivel: \blControlling, \blInfluencing.
	\BRItem[Versión] 1.0.
	\BRItem[Estado] Propuesta.
	\BRItem[Propuesta por] Ángeles
	\BRItem[Revisada por] Pendiente.
	\BRItem[Aprobada por] Pendiente.
	\BRItem[Descripción] La fecha de inicio de un periodo de recepción de solicitudes debe ser mayor a la fecha de temino.
	\BRItem[Sentencia]  $Fecha de Inicio < Fecha de Ternimo$
%	\BRItem[Motivación] Evitar que se defina un periodo de recepción de solicitudes de unidades de aprendizaje en un tiempo posterior a la planeación de la estructura educativa.
%	\BRItem[Ejemplo positivo] \cdtEmpty
%	\begin{itemize}
%		\item 
%		\item 
%		\item 
%	\end{itemize}
%	\BRItem[Ejemplo negativo]
%	\begin{itemize}
%		\item 
%		\item 
%		\item 
%	\end{itemize}
%	\BRItem[Referenciado por] \refIdElem{UA-CU2.1}
\end{BusinessRule}

%%======================================================================
\begin{BusinessRule}{BR-N027}{Solicitud válida}
	{\bcCondition}    % Clase: \bcCondition,   \bcIntegridad, \bcAutorization, \bcDerivation.
	{\btEnabler}     % Tipo:  \btEnabler,     \btTimer,      \btExecutive.
	{\blControlling}    % Nivel: \blControlling, \blInfluencing.
	\BRItem[Versión] 1.0.
	\BRItem[Estado] Propuesta.
	\BRItem[Propuesta por] Ángeles
	\BRItem[Revisada por] Pendiente.
	\BRItem[Aprobada por] Pendiente.
	\BRItem[Descripción] Una solicitud de unidades de aprendizaje es válida, sí y solo sí tiene asociada en cada unidad académica de adscripción del profesor por lo menos una unidad de aprendizaje.
%	\BRItem[Sentencia]  $Fecha de Inicio < Fecha de Ternimo$
%	\BRItem[Motivación] Evitar que se defina un periodo de recepción de solicitudes de unidades de aprendizaje en un tiempo posterior a la planeación de la estructura educativa.
%	\BRItem[Ejemplo positivo] \cdtEmpty
%	\begin{itemize}
%		\item 
%		\item 
%		\item 
%	\end{itemize}
%	\BRItem[Ejemplo negativo]
%	\begin{itemize}
%		\item 
%		\item 
%		\item 
%	\end{itemize}
%	\BRItem[Referenciado por] \refIdElem{UA-CU2.1}
\end{BusinessRule}

%%======================================================================
\begin{BusinessRule}{BR-N028}{Número de periodos anteriores para consulta}
	{\bcCondition}    % Clase: \bcCondition,   \bcIntegridad, \bcAutorization, \bcDerivation.
	{\btEnabler}     % Tipo:  \btEnabler,     \btTimer,      \btExecutive.
	{\blControlling}    % Nivel: \blControlling, \blInfluencing.
	\BRItem[Versión] 1.0.
	\BRItem[Estado] Propuesta.
	\BRItem[Propuesta por] Ángeles
	\BRItem[Revisada por] Pendiente.
	\BRItem[Aprobada por] Pendiente.
	\BRItem[Descripción] La información para consulta se aplica para los 4 periodos anteriores al periodo que se esta diseñando.
%	\BRItem[Sentencia]  $Fecha de Inicio < Fecha de Ternimo$
%	\BRItem[Motivación] Evitar que se defina un periodo de recepción de solicitudes de unidades de aprendizaje en un tiempo posterior a la planeación de la estructura educativa.
%	\BRItem[Ejemplo positivo] \cdtEmpty
%	\begin{itemize}
%		\item 
%		\item 
%		\item 
%	\end{itemize}
%	\BRItem[Ejemplo negativo]
%	\begin{itemize}
%		\item 
%		\item 
%		\item 
%	\end{itemize}
%	\BRItem[Referenciado por] \refIdElem{UA-CU2.1}
\end{BusinessRule}

%%======================================================================
\begin{BusinessRule}{BR-N029}{Promedio de ocupabilidad de unidad de aprendizaje}
	{\bcCondition}    % Clase: \bcCondition,   \bcIntegridad, \bcAutorization, \bcDerivation.
	{\btEnabler}     % Tipo:  \btEnabler,     \btTimer,      \btExecutive.
	{\blControlling}    % Nivel: \blControlling, \blInfluencing.
	\BRItem[Versión] 1.0.
	\BRItem[Estado] Propuesta.
	\BRItem[Propuesta por] David
	\BRItem[Revisada por] Pendiente.
	\BRItem[Aprobada por] Pendiente.
	\BRItem[Descripción] El promedio de ocupabilidad de una unidad de aprendizaje es igual a la suma de alumnos inscritos de los grupos asociados a la unidad de aprendizaje, entre el total de grupos asociados a la unidad de aprendizaje
	\BRItem[Sentencia]  $\dfrac{\sum^{\infty}_{n=0}NumeroTotalDeAlumnos}{NumeroDeGrupos}$
	\BRItem[Motivación] Definir la ocupabilidad de alumnos por unidad de aprendizaje.
%	\BRItem[Ejemplo positivo] \cdtEmpty
%	\begin{itemize}
%		\item 
%		\item 
%		\item 
%	\end{itemize}
%	\BRItem[Ejemplo negativo]
%	\begin{itemize}
%		\item 
%		\item 
%		\item 
%	\end{itemize}
%	\BRItem[Referenciado por] \refIdElem{UA-CU2.1}
\end{BusinessRule}

%%======================================================================
\begin{BusinessRule}{BR-N030}{Relación entre turnos}
	{\bcCondition}    % Clase: \bcCondition,   \bcIntegridad, \bcAutorization, \bcDerivation.
	{\btEnabler}     % Tipo:  \btEnabler,     \btTimer,      \btExecutive.
	{\blControlling}    % Nivel: \blControlling, \blInfluencing.
	\BRItem[Versión] 1.0.
	\BRItem[Estado] Propuesta.
	\BRItem[Propuesta por] Ángeles
	\BRItem[Revisada por] Pendiente.
	\BRItem[Aprobada por] Pendiente.
	\BRItem[Descripción] Debe existir al menos un turno definido entre matutino, vespertino y mixto.  Matutino y vespertino no se deben traslapar, mixto puede traslaparse con uno o con los dos.
	\BRItem[Sentencia]	\cdtEmpty
	\begin{itemize}
		\item $\forall Turnos \ni ProgramaAcademico \exists Turno Definido$
		\item $ HoraFinTurno1 = HoraInicioTurno2$
		\item $ Turno matutino \neq Turno Vespertino $
		\item $ HoraInicioTurno1 \neq HoraFinTurno1 $
	\end{itemize}
	\BRItem[Motivación] Evitar que se defina una Estructura Educativa sin turnos definidos para grupos así como evitar que el turno matutino y turno vespertino se traslapen.
	\BRItem[Ejemplo positivo 1] \cdtEmpty
	\begin{itemize}
			\item Turno matutino: 07:00-13:00
			\item Turno mixto: 12:00-15:00
			\item Turno vespertino: 15:00-22:00
	\end{itemize}
	\BRItem[Ejemplo positivo 2] \cdtEmpty
	\begin{itemize}
		\item Turno matutino: 07:00-13:00
	\end{itemize}
		\BRItem[Ejemplo positivo 3] \cdtEmpty
	\begin{itemize}
		\item Turno matutino: 07:00-13:00
		\item Turno mixto: 12:00-15:00
		\item Turno vespertino: 13:00-22:00
	\end{itemize}	
	\BRItem[Ejemplo negativo 1] \cdtEmpty
		\begin{itemize}
			\item Turno matutino: 07:00am - 02:00pm
			\item Turno matutino: 12:00 - 12:00
		\end{itemize}
		\BRItem[Ejemplo negativo 2] \cdtEmpty
	\begin{itemize}
		\item Turno matutino: \_\_:\_\_ - \_\_:\_\_
		\item Turno mixto: \_\_:\_\_ - \_\_:\_\_
		\item Turno Vespertino: \_\_:\_\_ - \_\_:\_\_
	\end{itemize}
	%	\BRItem[Referenciado por] \refIdElem{UA-CU2.1}
\end{BusinessRule}

%%======================================================================
\begin{BusinessRule}{BR-N031}{Relación de horas en turnos}
	{\bcCondition}    % Clase: \bcCondition,   \bcIntegridad, \bcAutorization, \bcDerivation.
	{\btEnabler}     % Tipo:  \btEnabler,     \btTimer,      \btExecutive.
	{\blControlling}    % Nivel: \blControlling, \blInfluencing.
	\BRItem[Versión] 1.0.
	\BRItem[Estado] Propuesta.
	\BRItem[Propuesta por] Ángeles
	\BRItem[Revisada por] Pendiente.
	\BRItem[Aprobada por] Pendiente.
	\BRItem[Descripción] hora de inicio debe ser menor que la hora de fin, el inicio de un turno puede ser el igual a la hora de fin de otro
	\BRItem[Sentencia]  $Hora de Inicio < Hora de Ternimo$
		\BRItem[Motivación] Evitar que se defina un turno con la hora de inicio mayor o igual a la hora de fin.
		\BRItem[Ejemplo positivo] \cdtEmpty
		\begin{itemize}
			\item Turno matutino: 07:00-15:00
	%		\item 
	%		\item 
		\end{itemize}
		\BRItem[Ejemplo negativo] \cdtEmpty
		\begin{itemize}
			\item Turno matutino: 14:00 - 12:00
	%		\item 
	 	\end{itemize}
	%	\BRItem[Referenciado por] \refIdElem{UA-CU2.1}
\end{BusinessRule}

%%======================================================================
\begin{BusinessRule}{BR-N032}{Ocupabilidad en grupo}
	{\bcCondition}    % Clase: \bcCondition,   \bcIntegridad, \bcAutorization, \bcDerivation.
	{\btEnabler}     % Tipo:  \btEnabler,     \btTimer,      \btExecutive.
	{\blControlling}    % Nivel: \blControlling, \blInfluencing.
	\BRItem[Versión] 1.0.
	\BRItem[Estado] Propuesta.
	\BRItem[Propuesta por] Ángeles
	\BRItem[Revisada por] Pendiente.
	\BRItem[Aprobada por] Pendiente.
	\BRItem[Descripción] La ocupabilidad de un grupo debe ser mayor que cero pero menor o igual X
	\BRItem[Sentencia]  $ \infty> ocupabilidad > 0$
	\BRItem[Motivación] Evitar que se definan grupos con capacidad igual o menor que cero.
		\BRItem[Ejemplo positivo] \cdtEmpty
		\begin{itemize}
			\item capacidad = 33
	%		\item 
	%		\item 
		\end{itemize}
		\BRItem[Ejemplo negativo] \cdtEmpty
		\begin{itemize}
			\item capacidad = -1
			\item capacidad = 0
	%		\item 
		\end{itemize}
	%	\BRItem[Referenciado por] \refIdElem{UA-CU2.1}
\end{BusinessRule}

%%======================================================================
\begin{BusinessRule}{BR-N033}{Definición de espacio}
	{\bcCondition}    % Clase: \bcCondition,   \bcIntegridad, \bcAutorization, \bcDerivation.
	{\btEnabler}     % Tipo:  \btEnabler,     \btTimer,      \btExecutive.
	{\blControlling}    % Nivel: \blControlling, \blInfluencing.
	\BRItem[Versión] 1.0.
	\BRItem[Estado] Propuesta.
	\BRItem[Propuesta por] Ángeles
	\BRItem[Revisada por] Pendiente.
	\BRItem[Aprobada por] Pendiente.
	\BRItem[Descripción] Una unidad de aprendizaje puede tener asociado un espacio
	%	\BRItem[Sentencia]  $Fecha de Inicio < Fecha de Ternimo$
	%	\BRItem[Motivación] Evitar que se defina un periodo de recepción de solicitudes de unidades de aprendizaje en un tiempo posterior a la planeación de la estructura educativa.
	%	\BRItem[Ejemplo positivo] \cdtEmpty
	%	\begin{itemize}
	%		\item 
	%		\item 
	%		\item 
	%	\end{itemize}
	%	\BRItem[Ejemplo negativo]
	%	\begin{itemize}
	%		\item 
	%		\item 
	%		\item 
	%	\end{itemize}
	%	\BRItem[Referenciado por] \refIdElem{UA-CU2.1}
\end{BusinessRule}

%%======================================================================
\begin{BusinessRule}{BR-N034}{Traslape de profesor}
	{\bcCondition}    % Clase: \bcCondition,   \bcIntegridad, \bcAutorization, \bcDerivation.
	{\btEnabler}     % Tipo:  \btEnabler,     \btTimer,      \btExecutive.
	{\blControlling}    % Nivel: \blControlling, \blInfluencing.
	\BRItem[Versión] 1.0.
	\BRItem[Estado] Propuesta.
	\BRItem[Propuesta por] Cesar
	\BRItem[Revisada por] Pendiente.
	\BRItem[Aprobada por] Pendiente.
	\BRItem[Descripción] Verifica que el docente seleccionado no tenga asignado el horario que se le está definiendo para una Unidad de Aprendizaje
	%	\BRItem[Sentencia]  $Fecha de Inicio < Fecha de Ternimo$
	%	\BRItem[Motivación] Evitar que se defina un periodo de recepción de solicitudes de unidades de aprendizaje en un tiempo posterior a la planeación de la estructura educativa.
	%	\BRItem[Ejemplo positivo] \cdtEmpty
	%	\begin{itemize}
	%		\item 
	%		\item 
	%		\item 
	%	\end{itemize}
	%	\BRItem[Ejemplo negativo]
	%	\begin{itemize}
	%		\item 
	%		\item 
	%		\item 
	%	\end{itemize}
	%	\BRItem[Referenciado por] \refIdElem{UA-CU2.1}
\end{BusinessRule}

%%======================================================================
\begin{BusinessRule}{BR-N035}{Carga Máxima}
	{\bcCondition}    % Clase: \bcCondition,   \bcIntegridad, \bcAutorization, \bcDerivation.
	{\btEnabler}     % Tipo:  \btEnabler,     \btTimer,      \btExecutive.
	{\blControlling}    % Nivel: \blControlling, \blInfluencing.
	\BRItem[Versión] 1.0.
	\BRItem[Estado] Propuesta.
	\BRItem[Propuesta por] Cesar
	\BRItem[Revisada por] Pendiente.
	\BRItem[Aprobada por] Pendiente.
	\BRItem[Descripción] La carga máxima de un profesor debe ser igual a la definida en su dictamen de categoría y a su número de horas basificadas
		\BRItem[Sentencia]  $Carga maxima = carga definida en dictamen de categoria \wedge numero de horas basificadas$
		\BRItem[Motivación] Verifica que le profesor cubra con la carga máxima que tiene asignada según su dictamen de categoría.
	%	\BRItem[Ejemplo positivo] \cdtEmpty
	%	\begin{itemize}
	%		\item 
	%		\item 
	%		\item 
	%	\end{itemize}
	%	\BRItem[Ejemplo negativo]
	%	\begin{itemize}
	%		\item 
	%		\item 
	%		\item 
	%	\end{itemize}
	%	\BRItem[Referenciado por] \refIdElem{UA-CU2.1}
\end{BusinessRule}

%%======================================================================
\begin{BusinessRule}{BR-N036}{Horario Laboral para Asignación de profesores a Unidades de Aprendizaje}
	{\bcCondition}    % Clase: \bcCondition,   \bcIntegridad, \bcAutorization, \bcDerivation.
	{\btEnabler}     % Tipo:  \btEnabler,     \btTimer,      \btExecutive.
	{\blControlling}    % Nivel: \blControlling, \blInfluencing.
	\BRItem[Versión] 1.0.
	\BRItem[Estado] Propuesta.
	\BRItem[Propuesta por] Cesar  
	\BRItem[Revisada por] Pendiente.
	\BRItem[Aprobada por] Pendiente.
	\BRItem[Descripción] La asignación de un profesor a la unidad de aprendizaje será válida si y solo si horario de la unidad de aprendizaje está dentro del horario laboral del profesor
	\BRItem[Sentencia]  $HorarioDeUnidadDeAprendizaje \in HorarioLaboral de Profesor$
		\BRItem[Motivación] Verifica que el horario que se le asigna a un profesor para una materia este dentro del horario laboral del profesor.
	%	\BRItem[Ejemplo positivo] \cdtEmpty
	%	\begin{itemize}
	%		\item 
	%		\item 
	%		\item 
	%	\end{itemize}
	%	\BRItem[Ejemplo negativo]
	%	\begin{itemize}
	%		\item 
	%		\item 
	%		\item 
	%	\end{itemize}
	%	\BRItem[Referenciado por] \refIdElem{UA-CU2.1}
\end{BusinessRule}

%%======================================================================

%%======================================================================
\begin{BusinessRule}{BR-N037}{Porcentaje de calificación}
	{\bcCondition}    % Clase: \bcCondition,   \bcIntegridad, \bcAutorization, \bcDerivation.
	{\btEnabler}     % Tipo:  \btEnabler,     \btTimer,      \btExecutive.
	{\blControlling}    % Nivel: \blControlling, \blInfluencing.
	\BRItem[Versión] 1.0.
	\BRItem[Estado] Propuesta.
	\BRItem[Propuesta por] Cesar
	\BRItem[Revisada por] Pendiente.
	\BRItem[Aprobada por] Pendiente.
	\BRItem[Descripción] La sumatoria de la calificación definida para la Unidad de Aprendizaje entre clases teóricas y prácticas debe ser igual a 100
	\BRItem[Sentencia]  $\dfrac{\sum^{\infty}_{n=0}Calificacion}{ClaseTeorica \wedge ClasePractica} = 100$
	\BRItem[Motivación] Verifica que la calificación o definida para la Unidad de Aprendizaje por tipo de clase sea igual que 100
		\BRItem[Ejemplo positivo] \cdtEmpty
		\begin{itemize}
			\item Clase Practica: 50\%
			\item Clase Teórica: 50\%
			\item Total de calificación: 100\% 
		\end{itemize}
		\BRItem[Ejemplo negativo] \cdtEmpty
		\begin{itemize}
			\item Clase Practica: 60\%
			\item Clase Teórica: 70\%
			\item Total de calificación: 130\%
		\end{itemize}
	%	\BRItem[Referenciado por] \refIdElem{UA-CU2.1}
\end{BusinessRule}

%%======================================================================
\begin{BusinessRule}{BR-N038}{Docente titular de teoría y práctica}
	{\bcCondition}    % Clase: \bcCondition,   \bcIntegridad, \bcAutorization, \bcDerivation.
	{\btEnabler}     % Tipo:  \btEnabler,     \btTimer,      \btExecutive.
	{\blControlling}    % Nivel: \blControlling, \blInfluencing.
	\BRItem[Versión] 1.0.
	\BRItem[Estado] Propuesta.
	\BRItem[Propuesta por] Cesar
	\BRItem[Revisada por] Pendiente.
	\BRItem[Aprobada por] Pendiente.
	\BRItem[Descripción] El docente que ha sido asignado como titular en la clase teórica deberá ser titular de al menos una unidad de aprendizaje en clase práctica
	%	\BRItem[Sentencia]  $Fecha de Inicio < Fecha de Ternimo$
	\BRItem[Motivación] Verifica que un profesor que sea titular de una clase de laboratorio sea por lo menos titular de una clase teórica
	%	\BRItem[Ejemplo positivo] \cdtEmpty
	%	\begin{itemize}
	%		\item 
	%		\item 
	%		\item 
	%	\end{itemize}
	%	\BRItem[Ejemplo negativo]
	%	\begin{itemize}
	%		\item 
	%		\item 
	%		\item 
	%	\end{itemize}
	%	\BRItem[Referenciado por] \refIdElem{UA-CU2.1}
\end{BusinessRule}

%%======================================================================
\begin{BusinessRule}{BR-N039}{Métricas de carga académica}
	{\bcCondition}    % Clase: \bcCondition,   \bcIntegridad, \bcAutorization, \bcDerivation.
	{\btEnabler}     % Tipo:  \btEnabler,     \btTimer,      \btExecutive.
	{\blControlling}    % Nivel: \blControlling, \blInfluencing.
	\BRItem[Versión] 1.0.
	\BRItem[Estado] Propuesta.
	\BRItem[Propuesta por] Cesar
	\BRItem[Revisada por] Pendiente.
	\BRItem[Aprobada por] Pendiente.
	\BRItem[Descripción] La carga académica tiene 4 métricas:
	\begin{itemize}
		\item El Director de la Unidad Académica y el o los docentes del sindicato pueden estar exentos de carga académica.
		\item Todo el personal docente que pertenezca al organigrama administrativo puede estar exento de carga máxima pero deberá cumplir la carga mínima.
		\item El personal docente restante tendrá la obligación de cumplir con su carga máxima.
		\item El personal técnico docente no debe tener carga frente a grupo.
	\end{itemize}
	%	\BRItem[Sentencia]  $Fecha de Inicio < Fecha de Ternimo$
	\BRItem[Motivación] Verificar que el personal cubra la carga correspondiente según lo establecido dentro de las métricas .
	%	\BRItem[Ejemplo positivo] \cdtEmpty
	%	\begin{itemize}
	%		\item 
	%		\item 
	%		\item 
	%	\end{itemize}
	%	\BRItem[Ejemplo negativo]
	%	\begin{itemize}
	%		\item 
	%		\item 
	%		\item 
	%	\end{itemize}
	%	\BRItem[Referenciado por] \refIdElem{UA-CU2.1}
\end{BusinessRule}

%%======================================================================
\begin{BusinessRule}{BR-N040}{Requisitos para asignación de espacios y/o profesores}
	{\bcCondition}    % Clase: \bcCondition,   \bcIntegridad, \bcAutorization, \bcDerivation.
	{\btEnabler}     % Tipo:  \btEnabler,     \btTimer,      \btExecutive.
	{\blControlling}    % Nivel: \blControlling, \blInfluencing.
	\BRItem[Versión] 1.0.
	\BRItem[Estado] Propuesta.
	\BRItem[Propuesta por] Cesar
	\BRItem[Revisada por] Pendiente.
	\BRItem[Aprobada por] Pendiente.
	\BRItem[Descripción] Los botones para asignar espacios y profesores estarán deshabilitados hasta que se cumpla la siguiente condición:\\
		\begin{itemize}
			\item Se podrán asignar espacios y/o profesores si y solo si ya se ha asignado el horario parcial o total a por lo menos un grupo del semestre seleccionado.
		\end{itemize}
	%	\BRItem[Sentencia]  $Fecha de Inicio < Fecha de Ternimo$
	\BRItem[Motivación] Evitar que se defina espacios y/o profesores si antes definir el horario total o parcial de por lo menos un grupo del semestre seleccionado
	%	\BRItem[Ejemplo positivo] \cdtEmpty
	%	\begin{itemize}
	%		\item 
	%		\item 
	%		\item 
	%	\end{itemize}
	%	\BRItem[Ejemplo negativo]
	%	\begin{itemize}
	%		\item 
	%		\item 
	%		\item 
	%	\end{itemize}
	%	\BRItem[Referenciado por] \refIdElem{UA-CU2.1}
\end{BusinessRule}

%%======================================================================
\begin{BusinessRule}{BR-N041}{Requisitos para asignación de unidades de aprendizaje}
	{\bcCondition}    % Clase: \bcCondition,   \bcIntegridad, \bcAutorization, \bcDerivation.
	{\btEnabler}     % Tipo:  \btEnabler,     \btTimer,      \btExecutive.
	{\blControlling}    % Nivel: \blControlling, \blInfluencing.
	\BRItem[Versión] 1.0.
	\BRItem[Estado] Propuesta.
	\BRItem[Propuesta por] Cesar
	\BRItem[Revisada por] Pendiente.
	\BRItem[Aprobada por] Pendiente.
	\BRItem[Descripción] El botón para asignar unidades de aprendizaje estará deshabilitado hasta que se cumpla la siguiente condición:\\
		\begin{itemize}
			\item Se podrán asignar unidades de aprendizaje si y solo si ya se ha terminado de definir los grupos del semestre seleccionado .
		\end{itemize}
	%	\BRItem[Sentencia]  $Fecha de Inicio < Fecha de Ternimo$
	\BRItem[Motivación] Evitar que se definan unidades de aprendizaje sin antes definir los grupos de una estructura educativa.
	%	\BRItem[Ejemplo positivo] \cdtEmpty
	%	\begin{itemize}
	%		\item 
	%		\item 
	%		\item 
	%	\end{itemize}
	%	\BRItem[Ejemplo negativo]
	%	\begin{itemize}
	%		\item 
	%		\item 
	%		\item 
	%	\end{itemize}
	%	\BRItem[Referenciado por] \refIdElem{UA-CU2.1}
\end{BusinessRule}
%%======================================================================
\begin{BusinessRule}{BR-N042}{Número de horas para unidades de aprendizaje}
	{\bcCondition}    % Clase: \bcCondition,   \bcIntegridad, \bcAutorization, \bcDerivation.
	{\btEnabler}     % Tipo:  \btEnabler,     \btTimer,      \btExecutive.
	{\blControlling}    % Nivel: \blControlling, \blInfluencing.
	\BRItem[Versión] 1.0.
	\BRItem[Estado] Propuesta.
	\BRItem[Propuesta por] Cesar
	\BRItem[Revisada por] Pendiente.
	\BRItem[Aprobada por] Pendiente.
	\BRItem[Descripción] El número de horas por definir para una unidad de aprendizaje debe ser igual al que tiene definido en el plan de estudios.
	\BRItem[Sentencia]  $NumeroDeHorasDeUnidadDeAprendizajeEnHorario = NumeroDEHorasUnidadDeAprendizajeEnPlanDeEstudios$
	\BRItem[Motivación] Verifica que el número de horas que se define para una materia en un grupo sea igual al número de horas que tiene definido en el plan de estudiosj.
	%	\BRItem[Ejemplo positivo] \cdtEmpty
	%	\begin{itemize}
	%		\item 
	%		\item 
	%		\item 
	%	\end{itemize}
	%	\BRItem[Ejemplo negativo]
	%	\begin{itemize}
	%		\item 
	%		\item 
	%		\item 
	%	\end{itemize}
	%	\BRItem[Referenciado por] \refIdElem{UA-CU2.1}
\end{BusinessRule}

%%======================================================================
\begin{BusinessRule}{BR-N043}{Traslape de horario}
	{\bcCondition}    % Clase: \bcCondition,   \bcIntegridad, \bcAutorization, \bcDerivation.
	{\btEnabler}     % Tipo:  \btEnabler,     \btTimer,      \btExecutive.
	{\blControlling}    % Nivel: \blControlling, \blInfluencing.
	\BRItem[Versión] 1.0.
	\BRItem[Estado] Propuesta.
	\BRItem[Propuesta por] Cesar
	\BRItem[Revisada por] Pendiente.
	\BRItem[Aprobada por] Pendiente.
	\BRItem[Descripción] El horario de una unidad de aprendizaje no puede ser el mismo para dos o mas unidades de aprendizaje en el mismo día.
	%	\BRItem[Sentencia]  $Fecha de Inicio < Fecha de Ternimo$
		\BRItem[Motivación] Verifica que el horario no haya sido asignado a otra unidad de aprendizaje en el mismo día.
	%	\BRItem[Ejemplo positivo] \cdtEmpty
	%	\begin{itemize}
	%		\item 
	%		\item 
	%		\item 
	%	\end{itemize}
	%	\BRItem[Ejemplo negativo]
	%	\begin{itemize}
	%		\item 
	%		\item 
	%		\item 
	%	\end{itemize}
	%	\BRItem[Referenciado por] \refIdElem{UA-CU2.1}
\end{BusinessRule}

%%======================================================================
\begin{BusinessRule}{BR-N044}{Rango de horario}
	{\bcCondition}    % Clase: \bcCondition,   \bcIntegridad, \bcAutorization, \bcDerivation.
	{\btEnabler}     % Tipo:  \btEnabler,     \btTimer,      \btExecutive.
	{\blControlling}    % Nivel: \blControlling, \blInfluencing.
	\BRItem[Versión] 1.0.
	\BRItem[Estado] Propuesta.
	\BRItem[Propuesta por] Cesar
	\BRItem[Revisada por] Pendiente.
	\BRItem[Aprobada por] Pendiente.
	\BRItem[Descripción] Los horarios de una unidad de aprendizaje deberán estar dentro del horario del turno al cuál pertenece el grupo en la que ha sido asignada.
	\BRItem[Sentencia]  $HorarioDeUnaUnidadDeApredizaje \in HorarioDeTurnoDefinidoParaGrupo$
	\BRItem[Motivación] Verifica que los horarios de una unidad de aprendizaje pertenezcan al horario del turno al que pertenece el grupo.
	%	\BRItem[Ejemplo positivo] \cdtEmpty
	%	\begin{itemize}
	%		\item 
	%		\item 
	%		\item 
	%	\end{itemize}
	%	\BRItem[Ejemplo negativo]
	%	\begin{itemize}
	%		\item 
	%		\item 
	%		\item 
	%	\end{itemize}
	%	\BRItem[Referenciado por] \refIdElem{UA-CU2.1}
\end{BusinessRule}

%%======================================================================
\begin{BusinessRule}{BR-N045}{Formato de horario}
	{\bcCondition}    % Clase: \bcCondition,   \bcIntegridad, \bcAutorization, \bcDerivation.
	{\btEnabler}     % Tipo:  \btEnabler,     \btTimer,      \btExecutive.
	{\blControlling}    % Nivel: \blControlling, \blInfluencing.
	\BRItem[Versión] 1.0.
	\BRItem[Estado] Propuesta.
	\BRItem[Propuesta por] Cesar
	\BRItem[Revisada por] Pendiente.
	\BRItem[Aprobada por] Pendiente.
	\BRItem[Descripción] El Formato de horario deberá ser ingresado en formato 24 horas y deberá ser de la siguiente manera:
		\begin{itemize}
			\item  \_ \_ : \_ \_ - \_ \_ : \_ \_
		\end{itemize}
	%	\BRItem[Sentencia]  $Fecha de Inicio < Fecha de Ternimo$
		\BRItem[Motivación] Verifica que el formato del horario este en 24 horas y que respete el formato.
		\BRItem[Ejemplo positivo] \cdtEmpty
		\begin{itemize}
			\item 07:00-12:00
	%		\item 
	%		\item 
		\end{itemize}
		\BRItem[Ejemplo negativo] \cdtEmpty
		\begin{itemize}
			\item 07:00/12:\_ \_
	%		\item 
	%		\item 
		\end{itemize}
	%	\BRItem[Referenciado por] \refIdElem{UA-CU2.1}
\end{BusinessRule}

